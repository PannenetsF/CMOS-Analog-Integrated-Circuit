\documentclass[lang=cn,11pt,a4paper,cite=authoryear]{elegantpaper}

% 微分号
\newcommand{\dd}[1]{\mathrm{d}#1}
\newcommand{\pp}[1]{\partial{}#1}

\newcommand{\homep}[1]{\section*{Problem #1}}

% FT LT ZT
\newcommand{\ft}[1]{\mathscr{F}[#1]}
\newcommand{\fta}{\xrightarrow{\mathscr{F}}}
\newcommand{\lt}[1]{\mathscr{L}[#1]}
\newcommand{\lta}{\xrightarrow{\mathscr{L}}}
\newcommand{\zt}[1]{\mathscr{Z}[#1]}
\newcommand{\zta}{\xrightarrow{\mathscr{Z}}}

% 积分求和号

\newcommand{\dsum}{\displaystyle\sum}
\newcommand{\aint}{\int_{-\infty}^{+\infty}}

% 简易图片插入
\newcommand{\qfig}[3][nolabel]{
  \begin{figure}[!htb]
      \centering
      \includegraphics[width=0.6\textwidth]{#2}
      \caption{#3}
      \label{\chapname :#1}
  \end{figure}
}

% 表格
\renewcommand\arraystretch{1.5}


% 日期


\title{CMOS模拟集成电路原理\quad 第十一周作业}
\author{范云潜 18373486}
\institute{微电子学院 184111 班}
\date{\zhtoday}

\begin{document}

\maketitle

% 作业内容:

\tableofcontents

\listoffigures

\section{流程简述}

SAR-ADC (Successive-approximation Register ADC) 是利用 DAC 模块逐次二分逼近最终得到合适的 ADC 值的一种方法。接下来简要介绍整体流程。

\begin{enumerate}
    \item 在一次转换的开始,如 \figref{01} ,对输入信号进行采样,所有由 DAC 输出控制的开关摆向底板,底板连接到输入信号,采样端此时需要接地,那么电容两端的电势差为 \(0 - V_{in}\) 。
    \item 采样完成后,需要将得到的信号进行保持,如 \figref{02} ,将所有由 DAC 控制的开关摆向地,将底板接到参考电压,采样端断开,此时电容的通路断开,不能放电,将所有的电荷保持住,采样的电压被拉到 \(- V_{in}\) ,如 \figref{04} 。
    \item 接下来进行逐次的比较,从最高位开始进行比较,如 \figref{03} ,将 DAC 输出的最高位置为高,此时的采样电压为 \(V_{ref} 2^{-1} - V_{in }\) ,比较 DAC 对应的电压与采样的电压的关系,若是不足则保持本位的输出,反之,取消本位。
    \item 对其他位依次进行以上操作,直到比较完成,向外部发出中断,标志本次转换已经完成,如 \figref{05} 。
\end{enumerate}


\qfig[01]{hw11p1.png}{对输入信号进行采样}

\qfig[02]{hw11p2.png}{对输入信号进行保持}

\qfig[03]{hw11p3.png}{逐次逼近信号大小}

\qfig[04]{hw11p4.png}{断开回路的电容采样电压}

\qfig[05]{hw11p5.png}{转换完成后输出中断信号}

\section{逼近器设计}

首先,作为一个时序逻辑数字模块,需要存在某种形式的时钟信号与复位信号,实际使用的模块还需要使用使能信号来启用模块以及启动信号。另外,根据上一节的分析,需要使用采样/保持信号,以及比较结果作为输入进行比较,需要每位控制开关的数据信号\footnote{同时也是系统的输出}。据此定义模块的端口。

\begin{lstlisting}
module sar (
    clk,
    reset,
    en,
    start,
    intr,
    hold,
    cmp,
    data
);

// the timing signal and control signal 
input clk, reset, en, start;
// the interrupt and discharge 
output reg intr, hold;

// the result of comparator 
input cmp;
// the output of SAR 
output reg [4:0] data;
\end{lstlisting}

接下来定义各个状态,状态之间的转换关系如下:

\begin{itemize}
    \item \lstinline{idle}:等待开始信号,之后转换到 \lstinline{init} 。此时,不进行保持。
    \item \lstinline{init}:DAC 输出的信号为全高,直接转换到 \lstinline{ready}。
    \item \lstinline{ready}:此时开始保持,进行采样,进入 \lstinline{compare} 。
    \item \lstinline{compare}:修改输出的数据,进入\lstinline{check} 。
    \item \lstinline{check}:对来自比较器的数据进行检测,决定下一状态的数据。
\end{itemize}

\section{设计过程}

设计的数字模块如 \figref{08} ,设计的多路选择器如 \figref{09},模块整体如 \figref{10} ,封装后如 \figref{11},其配置文件如 \figref{12} 。

\qfig[08]{hw11p8.png}{SAR 模块接口}

\qfig[09]{hw11p9.png}{多路选择器内部设计}

\qfig[10]{hw11p10.png}{SAR ADC 整体设计}

\qfig[11]{hw11p11.png}{封装测试台示意图}

\qfig[12]{hw11p12.png}{数模混合配置}

\section{实验结果}

ADC 的输出如 \figref{06} ,间隔见 \figref{07} 其中可见运行的中断存在,中断间隔小于 \(1 us\) ,满足要求。

\qfig[06]{hw11p6.png}{ADC 采样输出波形}

\qfig[07]{hw11p7.png}{ADC 采样输出中断间隔}
% Start Here

% End Here

\end{document}