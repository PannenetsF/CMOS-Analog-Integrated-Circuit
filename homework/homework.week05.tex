\documentclass[lang=cn,11pt,a4paper,cite=authoryear]{elegantpaper}

% 微分号
\newcommand{\dd}[1]{\mathrm{d}#1}
\newcommand{\pp}[1]{\partial{}#1}

\newcommand{\homep}[1]{\section*{Problem #1}}

% FT LT ZT
\newcommand{\ft}[1]{\mathscr{F}[#1]}
\newcommand{\fta}{\xrightarrow{\mathscr{F}}}
\newcommand{\lt}[1]{\mathscr{L}[#1]}
\newcommand{\lta}{\xrightarrow{\mathscr{L}}}
\newcommand{\zt}[1]{\mathscr{Z}[#1]}
\newcommand{\zta}{\xrightarrow{\mathscr{Z}}}

% 积分求和号

\newcommand{\dsum}{\displaystyle\sum}
\newcommand{\aint}{\int_{-\infty}^{+\infty}}

% 简易图片插入
\newcommand{\qfig}[3][nolabel]{
  \begin{figure}[!htb]
      \centering
      \includegraphics[width=0.6\textwidth]{#2}
      \caption{#3}
      \label{\chapname :#1}
  \end{figure}
}

% 表格
\renewcommand\arraystretch{1.5}


% 日期


\title{CMOS模拟集成电路原理\quad 第五周作业}
\author{范云潜 18373486}
\institute{微电子学院 184111 班}
\date{\zhtoday}

\begin{document}

\maketitle


\tableofcontents

\listoffigures

\section{作业内容}

对一偏置电流为100uA的五管OTA,
共模电压为0.9V,设计晶体管的尺寸,
1. 使其在单位增益负反馈时系统性失调小
于0.1mV;

2. 使其随机性失调的标准差(std)小于
1mV;

3. 使其共模抑制比大于 50dB。

\qfig[hw]{hw05.png}{作业电路}

% Start Here


\section{系统性失调}

分析 \(\Delta V_{out}\) :

\[\Delta V_{out} = R_{eq, pmos} * \Delta I_B \]

% \[R_{eq, pmos} = \frac{\Delta I_{ds}}{\Delta V_{ds}} = \lambda_p I_{ds, p}\]

\[\Delta I_B = \lambda_n \Delta V_{ds, n} I\]

因此,需要降低 \(\lambda\) ,一般来说,沟道越长,\(\lambda\) 越小。所以需要尽量增大沟道长度 \(L\) 。


\section{随机性失调}

需要使得 \(V_{gs}-V_t\) 尽量小,因此同样电流下, \(W/L\) 更大。而之前一节已经确定需要增大 \(L\) ,因此 \(W\) 也需要更大。

\section{电路搭建}

对于已经给定的偏置电流 \(100 \mu A\) 可以确定各个晶体管通过的电流大小,通过预估 pMOS 与 nMOS 的阈值电压,设置各个晶体管的长宽比。长宽比的设置可以通过 Aether 自带的 DC 扫描计算得到,以 pMOS 为例,搭建电路如 \figref{01} ,将参数输入到 MDE 的设置中,如 \figref{02},得到电流扫描的图像,读取对应电流下的长宽比 \(k\) 即可。

\qfig[01]{hw05f1.png}{扫描 pMOS 长宽比电路}

\qfig[02]{hw05f2.png}{扫描 pMOS 长宽比 MDE 设置}

根据上述方法,预估好关键电压后,确定 OTA 的各个晶体管尺寸,对差分输入为 \(0 V\) 时进行 OP 仿真,电流满足需求,如 \figref{04} 。

\qfig[04]{hw05f4.png}{OTA (未进行单位增益缓冲)各管状态以及电流大小}


\section{系统性失调验证}

进一步,通过 cds\underline{ }thru 连接成单位增益负反馈电路,如\figref{08},对其输出进行 DC 仿真,其 \(V_{in}\) 与原有值的偏离即为系统性失调。经过仿真,如 \figref{09} ,为 \(97 \mu V\) ,满足要求。

\qfig[08]{hw05f8.png}{单位增益负反馈电路}

\qfig[09]{hw05f9.png}{系统性失调转换到输入端为 \(97 \mu V\)} 

\section{随机性失调验证}

将模型切换到专用的蒙特卡洛模型,并恢复为差分电路。在 DC 仿真下对输入电压进行 400 次随机测试,可以保证最坏在 \(2 \sigma\) 范围外,如 \figref{07} ,可以得到 \(2 \sigma < 800 \mu\) ,满足要求。

\qfig[07]{hw05f7.png}{蒙特卡洛仿真}

\section{CMRR 验证}

根据公式 \(\text{CMRR} = A_{v,d}/A_{v,c}\) ,分别对共模输入和差分信号造成的增益进行 AC 仿真,分别如 \figref{05},\figref{06},可得 \(A_{v,d} = 250, A_{v,d} = 0.13\) ,那么 \(\text{CMRR} = 250 / 0.13 = 1923\) ,转换为分贝数,\(10 \times \log 1923 = 75.616 DB\) ,满足要求。


\qfig[05]{hw05f5.png}{OTA 共模增益}

\qfig[06]{hw05f6.png}{OTA 差分增益}

\end{document}