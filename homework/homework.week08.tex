\documentclass[lang=cn,11pt,a4paper,cite=authoryear]{elegantpaper}

% 微分号
\newcommand{\dd}[1]{\mathrm{d}#1}
\newcommand{\pp}[1]{\partial{}#1}

\newcommand{\homep}[1]{\section*{Problem #1}}

% FT LT ZT
\newcommand{\ft}[1]{\mathscr{F}[#1]}
\newcommand{\fta}{\xrightarrow{\mathscr{F}}}
\newcommand{\lt}[1]{\mathscr{L}[#1]}
\newcommand{\lta}{\xrightarrow{\mathscr{L}}}
\newcommand{\zt}[1]{\mathscr{Z}[#1]}
\newcommand{\zta}{\xrightarrow{\mathscr{Z}}}

% 积分求和号

\newcommand{\dsum}{\displaystyle\sum}
\newcommand{\aint}{\int_{-\infty}^{+\infty}}

% 简易图片插入
\newcommand{\qfig}[3][nolabel]{
  \begin{figure}[!htb]
      \centering
      \includegraphics[width=0.6\textwidth]{#2}
      \caption{#3}
      \label{\chapname :#1}
  \end{figure}
}

% 表格
\renewcommand\arraystretch{1.5}


% 日期


\title{CMOS模拟集成电路原理\quad 第八周作业}
\author{范云潜 18373486}
\institute{微电子学院 184111 班}
\date{\zhtoday}

\begin{document}

\maketitle

作业内容:• 设计一轨到轨输入运放,要求指标:

• V DD =1.8V

• GBW=100MHz, C L =10pF

• 完成设计后给出以下参数:

• PM

• FOM

• 0.4V-1.4V区间内的系统失调电压

• 0.4V-1.4V区间内的GBW偏差

\tableofcontents

\listoffigures


\section{电路工作原理}

首先,为了平衡跨导,我们采用了三倍电流镜;其次为了使用该电流进行放大,使用简单电流镜负载,电路如 \figref{01} 。

当输入的电压使得 P1 或者 N1 其一截止时,为了维持输出的电流,将原应该从 P1 或 N1 中流过的电流,通过放大使得未截止的管子流过原本四倍的电流,以满足 \(g_{m,n} + g_{m,p} = \text{const}\) 。这也是三倍电流的来源。

将第一级的电路视作电流的提供者,这份电流实际上是提供了 \(2 g_m\) 的跨导,因此电路有 \(g_m \times 2 = 2 \pi C_L \cdot GBW \) 。

\qfig[01]{hw08p1.png}{基本电路}

\section{电路设计}

电路的设计如 \figref{02} ,采取了极度保守的设计,将目标增益带宽积设为 \(240 M\) ,来防止在输入极大或者极小的情况不满足预期。承载相同电流的管子的尺寸一致,得到的数据如下: 

\begin{lstlisting}
    L: um:  0.5
    In: uA:  753.9822368615504 
    Ip: uA:  753.9822368615504
    W-p: um  269.27937030769647 # 
    IB-p: um  269.27937030769647
    3*IB-p: um  807.8381109230894
    W-n: um  67.31984257692412
    IB-n: um  67.31984257692412
    3*IB-n: um  201.95952773077235
\end{lstlisting}

\qfig[02]{hw08p2.png}{仿真含参数电路搭建}

带入电路中,得到电路如 \figref{03} ,进一步封装。

\qfig[03]{hw08p3.png}{不含参数电路}



\section{性能测试} 

\subsection{增益带宽积与相位裕度}

首先进行差模输入,AC 测试其增益带宽积,测试电路如 \figref{04} ,测试参数如 \figref{05} ,结果如 \figref{06} ,增益带宽积为 \(270 M\) ,相位裕度 \(70^\circ\)。基本符合预期。

\qfig[04]{hw08p4.png}{封装后的测试电路}


\qfig[05]{hw08p5.png}{差模增益测试参数}


\qfig[06]{hw08p6.png}{差模增益}

\subsection{FOM} 

计算品质系数:

\[FOM = \frac{GBW \cdot C_L}{I_B} = \frac{270 M \cdot 10 p}{7 \cdot 754 u} = 0.51156\] 

\subsection{失调电压} 

为了测量失调的特性,需要对电路进行 Buffer 式接法,并且转换为 DC 仿真,电路如 \figref{07} ,其失调如 \figref{08} ,在 \(3 mV\) 内。

\qfig[07]{hw08p7.png}{单位增益接法电路}

\qfig[08]{hw08p8.png}{失调电压}

\subsection{增益带宽积偏差}

为了实现在同时进行多个直流状态的交流测试,需要在 AC 仿真内部开启蒙特卡洛仿真的高级选项,对直流参数进行扫描,如 \figref{09} 。


\qfig[09]{hw08p9.png}{蒙特卡洛仿真设置} 

最终得到仿真结果,如 \figref{10} ,\figref{11} 。增益带宽积偏差达到了 \(140 M\) ,但是最坏情况也满足要求。

\qfig[10]{hw08p10.png}{增益带宽积蒙特卡洛仿真}

\qfig[11]{hw08p11.png}{增益带宽积偏差}
% Start Here
\section{调节过程}

本次电路没有经历任何的尺寸调节,但是对于参考电压 \(V_{r,p/n} \) 需要注意,在其造成的对应晶体管 \(V_{GST}\) 过大时,在输入为 \(0 V\) 或者 \(1.8 V\) 时,电流仍不能被完全抽取,因此应该适当调小其 \(V_{GST}\) 。

% End Here

\end{document}