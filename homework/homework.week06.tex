\documentclass[lang=cn,11pt,a4paper,cite=authoryear]{elegantpaper}

% 微分号
\newcommand{\dd}[1]{\mathrm{d}#1}
\newcommand{\pp}[1]{\partial{}#1}

\newcommand{\homep}[1]{\section*{Problem #1}}

% FT LT ZT
\newcommand{\ft}[1]{\mathscr{F}[#1]}
\newcommand{\fta}{\xrightarrow{\mathscr{F}}}
\newcommand{\lt}[1]{\mathscr{L}[#1]}
\newcommand{\lta}{\xrightarrow{\mathscr{L}}}
\newcommand{\zt}[1]{\mathscr{Z}[#1]}
\newcommand{\zta}{\xrightarrow{\mathscr{Z}}}

% 积分求和号

\newcommand{\dsum}{\displaystyle\sum}
\newcommand{\aint}{\int_{-\infty}^{+\infty}}

% 简易图片插入
\newcommand{\qfig}[3][nolabel]{
  \begin{figure}[!htb]
      \centering
      \includegraphics[width=0.6\textwidth]{#2}
      \caption{#3}
      \label{\chapname :#1}
  \end{figure}
}

% 表格
\renewcommand\arraystretch{1.5}


% 日期


\title{CMOS模拟集成电路原理\quad 第六周作业}
\author{范云潜 18373486}
\institute{微电子学院 184111 班}
\date{\zhtoday}


\begin{document}

\maketitle

作业内容:已知要求GBW=50MHz,C L =5pF。设计一相位裕度大于70°的米勒运放。通过仿真结果,给出该设计运放的:
1. FOM
2. 相位裕度
3. 输入等效总噪声

\tableofcontents

\listoffigures

\section{基本参数计算}

接下来将根据一系列公式进行计算。首先,保证输出级特征频率 \(f_t = \alpha \gamma (1 + \beta) \cdot GBW\) ,补偿电容和负载电容满足 \(C_c = C_L / \alpha\) , 寄生电容满足 \(C_{gs6} = C_C / \beta\) ,那么 \(g_{m6} = f_t \cdot 2 \pi C_{gs6} = 0.00209\) ,\(g_{m1} = 2 \pi C_C GBW\) 。估算长度,暂时取 \(n=1\) ,\(L = \sqrt{2 \mu V_{gst6} / (4 \pi f_T)}\) 。估算宽度,\(W_6 = C_{gs6} / k\) 。电流按照 \(g_m / 10\) 估算。为了让 \(g_{m6}\) 充分大,并且 \(g_{m6} / g_{m1}\) 充分大,那么 \(\alpha, \gamma \) 也要充分大,最终尝试多次后,\(\alpha \beta \gamma = 6,3,6\) 可以满足条件。

\begin{lstlisting}[caption={估算程序}]
    import math 

    kn = 280e-6
    kp = 60e-6

    alpha = 6
    beta = 3
    gamma = 6
    vgst6 = 0.2
    vgst1 = 0.2
    vgst3 = 0.2

    mu = 33e-3

    gbw = 50e6
    cl = 5e-12

    ft = alpha * gamma * (1 + beta) * gbw 

    cc = cl / alpha 
    cgs6 = cc / beta 
    gm6 = ft * 2 * math.pi * cgs6
    gm1 = 2 * math.pi * cc * gbw 

    l = (3 * mu * vgst6 / (4 * math.pi * ft))**0.5

    k = 2e-15/1e-6
    w6 = cgs6 / k 
    i6 = gm6/2 * vgst6 

    i1 = gm1/10 

    w1 = 2 * i1 * l / kp / vgst1**2
    w3 = 2 * i1 * l / kn / vgst3**2

    print('l:u ', l*1e6)
    print('w6:u ', w6*1e6)
    print('w1:u ', w1*1e6)
    print('w3:u ', w3*1e6)
    print('cc ', cc)
    print('gm6 ', gm6)
    print('gm1 ', gm1)
    print('i6:u ', i6*1e6)
    print('i1:u ', i1*1e6)
    print('i6/i1: ', i6/i1)
    
\end{lstlisting}

\section{基本电路搭建}

一个基本的 OTA 电路填入计算的参数,计算剩余参数,完成偏置。

\qfig[01]{hw0601.png}{基本电路图}

最终确定共模电压 \(0.3V\) ,参考电流 \(26.1 uA\) 。电路如 \figref{01} ,参数如 \figref{03} \footnote{由于存在如 \(par(B)*par(Width)\) 的参数,软件显示为 “*Error*” ,但是可以正常仿真。}

\qfig[03]{hw0603.png}{基本参数设置}

\section{测试平台介绍}

首先进行 OP 仿真确定各个管子的工作状态,如 \figref{02} 和估计值几乎吻合,因此保留了下来。

\qfig[02]{hw0602.png}{各管状态}

接下来进行增益测试,如 \figref{04} ,\(GBW = 64 M\) ,\(PM = 72^\circ\) 。

\qfig[04]{hw0604.png}{增益波形}

\section{性能评估}

\subsection{\(FOM\)}

偏置电流如 \figref{05} ,\(FOM = GBW \cdot C_L / I_B = 64M 5 pF / (51.3 / 1000 mA) = 6237MHz pF/mA\) 。

\qfig[05]{hw0605.png}{偏置电流}

\subsection{相位裕度}

\(PM = 72^\circ\) 。

\subsection{输入等效噪声}

如 \tabref{tab:my-tabletab:my-table} 。
\begin{table}[]
    \caption{等效输入}
    \centering
    \label{tab:my-tabletab:my-table}
    \begin{tabular}{|c|c|c|c|c|c|}
    \hline
    Freq(Hz)                               & 1         & 10        & 100      & 1k         & \multicolumn{1}{c|}{10k}    \\ \hline
    Noise,eq,in(V/Hz\textasciicircum{}0.5) & 88.9926m  & 31.7779m  & 11.7971m & 4.4839m    & \multicolumn{1}{c|}{1.7283m} \\ \hline
    100k      & 1meg      & 10meg    & 19.9526meg &   39.8107meg    & 50.1187meg                     \\ \hline 
    675.9841u & 271.4872u & 40.9206u  & 13.6151u    &  4.9753u       & 3.7160u                   \\ \hline
    \end{tabular}
\end{table}

% Start Here

% End Here

\end{document}