\documentclass[lang=cn,11pt,a4paper,cite=authoryear]{elegantpaper}

% 微分号
\newcommand{\dd}[1]{\mathrm{d}#1}
\newcommand{\pp}[1]{\partial{}#1}

\newcommand{\homep}[1]{\section*{Problem #1}}

% FT LT ZT
\newcommand{\ft}[1]{\mathscr{F}[#1]}
\newcommand{\fta}{\xrightarrow{\mathscr{F}}}
\newcommand{\lt}[1]{\mathscr{L}[#1]}
\newcommand{\lta}{\xrightarrow{\mathscr{L}}}
\newcommand{\zt}[1]{\mathscr{Z}[#1]}
\newcommand{\zta}{\xrightarrow{\mathscr{Z}}}

% 积分求和号

\newcommand{\dsum}{\displaystyle\sum}
\newcommand{\aint}{\int_{-\infty}^{+\infty}}

% 简易图片插入
\newcommand{\qfig}[3][nolabel]{
  \begin{figure}[!htb]
      \centering
      \includegraphics[width=0.6\textwidth]{#2}
      \caption{#3}
      \label{\chapname :#1}
  \end{figure}
}

% 表格
\renewcommand\arraystretch{1.5}


% 日期


\title{CMOS模拟集成电路原理\quad 第一周作业}
\author{范云潜 18373486}
\institute{微电子学院 184111 班}
\date{\zhtoday}

\begin{document}

\maketitle

作业内容:已知要求GBW=50MHz,C L =5pF。设计一相位裕度大于70°的米勒运放。通过仿真结果,给出该设计运放的:
1. FOM
2. 相位裕度
3. 输入等效总噪声

\tableofcontents

\listoffigures



\section{基本参数计算}

接下来将根据一系列公式进行计算。首先,保证输出级特征频率 \(f_t = 16 \cdot GBW\) ,补偿电容和负载电容满足 \(C_c = C_L / \alpha\) , 寄生电容满足 \(C_{gs6} = C_C / \beta\) ,那么 \(g_{m6} = f_t \cdot 2 \pi C_{gs6} = 0.00209\) ,\(g_{m1} = 2 \pi C_C GBW = 0.000524\) 。估算长度,暂时取 \(n=1\) ,\(L = \sqrt{2 \mu V_{gst6} / (4 \pi f_T)} \approx 1.5 u\) 。估算宽度,\(W_6 = C_{gs6} / k = 208 u\) 。电流按照 \(g_m / 10\) 估算。

\section{基本电路搭建}

一个基本的 OTA 电路填入计算的参数,计算剩余参数,完成偏置。注意到 \(I_6/I_1 = 4\) ,那么 \(B = 2\) 。

\qfig[01]{hw0601.png}{基本电路图}

最终确定共模电压 \(0.6V\) ,参考电流 \(52.4 uA\) 。电路如 \figref{01} ,参数如 \figref{03} \footnote{由于存在如 \(par(B)*par(Width)\) 的参数,软件显示为 “*Error*” ,但是可以正常仿真。}

\qfig[03]{hw0603.png}{基本参数设置}

\section{测试平台介绍}

首先进行 OP 仿真确定各个管子的工作状态,如 \figref{02} ,但是第二级的线性区一直存在,诸次修改也无功而反,但是检查跨导之后和估计值几乎吻合,因此保留了下来。

\qfig[02]{hw0602.png}{各管状态}

接下来进行增益测试,如 \figref{04} ,\(GBW = 80 \cdot 6.2 M = 500 M\) ,\(PM = 78^\circ\) 。

\qfig[04]{hw0604.png}{增益波形}

\section{性能评估}

\subsection{\(FOM\)}

偏置电流如 \figref{05} ,\(FOM = GBW \cdot C_L / I_B = 500 M 5 pF / (38.64 / 1000 mA) = 64700 MHz pF/mA\) 

\qfig[05]{hw0605.png}{偏置电流}

\subsection{相位裕度}

\(PM = 78^\circ\) 。

\subsection{输入等效噪声}

如 \tabref{tab1} 。
\begin{table}[]
    \caption{等效输入}
    \centering
    \label{tab:my-tabletab:my-table}
    \begin{tabular}{|c|c|c|c|c|c}
    \hline
    Freq(Hz)                               & 1         & 10        & 100      & 1k         & \multicolumn{1}{c|}{10k}    \\ \hline
    Noise,eq,in(V/Hz\textasciicircum{}0.5) & 83.3787m  & 30.8746m  & 11.7278m & 4.5441m    & \multicolumn{1}{c|}{1.844m} \\ \hline
    Freq(Hz)                               & 100k      & 1meg      & 10meg    & 19.9726meg &                             \\ \cline{1-5}
    Noise,eq,in(V/Hz\textasciicircum{}0.5) & 872.4678u & 200.0779u & 5.1715u  & 1.9504u    &                             \\ \cline{1-5}
    \end{tabular}
\end{table}

% Start Here

% End Here

\end{document}