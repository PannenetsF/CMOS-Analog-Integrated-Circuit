\documentclass[lang=cn,11pt,a4paper,cite=authoryear]{elegantpaper}

% 微分号
\newcommand{\dd}[1]{\mathrm{d}#1}
\newcommand{\pp}[1]{\partial{}#1}

\newcommand{\homep}[1]{\section*{Problem #1}}

% FT LT ZT
\newcommand{\ft}[1]{\mathscr{F}[#1]}
\newcommand{\fta}{\xrightarrow{\mathscr{F}}}
\newcommand{\lt}[1]{\mathscr{L}[#1]}
\newcommand{\lta}{\xrightarrow{\mathscr{L}}}
\newcommand{\zt}[1]{\mathscr{Z}[#1]}
\newcommand{\zta}{\xrightarrow{\mathscr{Z}}}

% 积分求和号

\newcommand{\dsum}{\displaystyle\sum}
\newcommand{\aint}{\int_{-\infty}^{+\infty}}

% 简易图片插入
\newcommand{\qfig}[3][nolabel]{
  \begin{figure}[!htb]
      \centering
      \includegraphics[width=0.6\textwidth]{#2}
      \caption{#3}
      \label{\chapname :#1}
  \end{figure}
}

% 表格
\renewcommand\arraystretch{1.5}


% 日期


\title{CMOS模拟集成电路原理\quad 第十周作业}
\author{范云潜 18373486}
\institute{微电子学院 184111 班}
\date{\zhtoday}

\begin{document}

\maketitle


\tableofcontents

\listoffigures

% Start Here

\section{电路理解与分析}

本次实验电路如 \figref{01} ,进行以下几个模块的拆分:

\begin{itemize}
    \item 差分对: MN1+MI(1,2) 。输出差分电流
    \item 电流放大器: M(5-8, 11-14) 。 对输入的差分电流转化成电压输出。将 \(I_{+} - I_{-}\) 变化为 \(I_{in}\) 。MA(5-10) ,可看作电阻 \(R_{in}\),为 MA(3,4) 提供偏置,并且和输出作为线性跨导回路。
    \item 输出级: 通过 MA(1,2) 放大电压信号。
    %TODO: MN3 是个 NMOS? 
\end{itemize}



\qfig[01]{hw10p1.png}{线性跨导回路电路}


分析需要的放大倍数: 

\[\frac{10 A}{A + 11} = 9.5\]

解得 \(A = 209\) ,五倍冗余 \(A = 1045\) 。


这是一个两级的放大器,第一级是电压-电流放大,其大小为 \(g_{m,I}\) ,估算为 \(0.1 m\);第二级输入的电压是电流通过电阻转换成的,大小为 \(R_{in} = g_{m,9} r_{o,9} r_{o,10} \) 估算为 \(0.01 \times (1/\lambda I)^2 = 0.01 (10 / 10u)^2 = 10^{10} \) 。最后一级电流设为 \(0.5 m\) 。放大为 \(g_{m,o} R_L = 0.15\) 总体放大为 \(15 \times 10^4\) ,必然满足条件。

由于必须设计反馈电容,考虑主极点:\(g_{m,I} / 2 \pi C_c\) ,以及为了在 \(200 k\) 增益不下降时工作,留足裕度到 \(10 M\) ,选择 \(5 p\) ,\(1m / 7 5p  = 30 M\) ,此时 \(\alpha = 2\) 符合经验。

考虑静态功耗,设置输出级电流为 \(500 u\) ,输入级为 \(100 u \times 2\) ,线性跨导环为 \(5 u\) ,其他为 \(10 u\) 。

\section{搭建电路} 

基本电路图如 \figref{02} 。设置长度为 \(L = 1u\) ,接下来进行分析。

\qfig[02]{hw10p2.pdf}{基本电路图}



\subsection{差分对}

对于差分对来说,其输入共模为 \(0.9 V\) ,且按照估算,其电流需要有 \(100 u\) ,设其源极电压为 \(0.2 V\) ,漏极电压为 \(1.2V\),其下方  MN1 按照电流镜进行设置,为 \(200 u\) 。电流镜应尽量宽,来使得其消耗的电压较小,让差分对可以达到较高的栅源电压,计算得到 MI(1-2) 为 \(4 u\) 为了其驱动能力,增大三倍到 \(12 u\),MN1 在过驱动电压为 \(0.1 V\) 时为 \(40 u\) 。

\subsection{电流源负载} 

对于这样两端为电流镜负载,而中间需要连接到线性跨导环的并联两端,因此需要保持 MA(3-4) 上下分别作为最后一级的 PMOS 和 NMOS 栅极电压,限制到 \(1.2 V\) 和 \(0.6 V\) 左右。

按照之前分析过程,设置其静态电流为 \(10 u\) \footnote{最上面管子为 \(110 u\) 。},从上到下的 PMOS 过驱动电压分别设置为 \(0.15 V\) 、 \(0.15 V\) , 计算得到的宽为 \(28 u\) ,\(3 u\) ,同样地,考虑驱动能力,以及减少电压裕度的消耗, 分别扩展到 \(50 u\) , \(40 u\) 。使用电流镜实现。

NMOS 分析过程是类似的,但是需要注意到折叠电流镜的存在,下面这一路的电压消耗会尽量小,靠下的 M(5-6) 的栅极电压预设为 \(0.6 V\) 而 M(7-8) 设计为 \(0.8 V\) ,得到的尺寸为 \(1 u\) 和 \(2 u\) 。

\subsection{线性跨导环-非并联部分} 

在不考虑跨导环结构仅仅考虑静态工作点时,三个管子的结构是为了 MA(3-4) 提供偏置,因此需要其偏置点大约到 NMOS \(1 V\) , PMOS \(0.6 V\) ,此处将两个二极管接法的管子尺寸设为一致,将另一电流镜控制的管子尺寸设为响应的比例,计算得到二极管接法 PMOS 管子在 \(0.2 V\) 过驱动尺寸为 \(1 u\) , NMOS 管子为 \(1 u\) ,但是取得原始尺寸会存在电流镜处为前一级提供偏置电压不符合,因此将三个管子比例做适当调节,满足静态工作要求。

\subsection{线性跨导环-并联到电流镜负载部分} 

之前两个小节设定好了管子的三个电压,此时可以适当调节两个管子的电流比例,来满足在输出级的电压需求。

\subsection{输出级} 

将电流设置为线性跨导环的 \(120\) 倍,基本达到 \(500 u\) 的电流。

\subsection*{尺寸与偏置总结}

最终,得到的尺寸如 \figref{0304} 。但是由于 MA1 的输入电压太高,需要拉低,将 M(5-6) 调到了线性区,而这对增益影响较小,如 \figref{05} 。为了让输出级宽长比更大,将 \(L\) 改到 \(0.2 u\) 。

\begin{figure}
    \caption{尺寸参数}\label{0304}
    \centering
    \sfig[0.45]{hw10p3.png}
    \sfig[0.45]{hw10p4.png}
\end{figure}

\qfig[05]{hw10p5.png}{电路状态}

\section{电路性能}

\subsection{差分增益带宽积}

利用 VCVS 等产生输出差分信号,开启交流信号,输出负载为 \(32 \Omega\) 和 \(10 pF\) ,得到增益如 \figref{06} ,增益和增益带宽积为 \(80k\) 与 \(22 M\) ,电路分析的结果为 \(150 k\) 与 \(30 M\) 基本符合。

\qfig[06]{hw10p6.png}{增益带宽积}

\subsection{放大性能} 

在进行静态功耗的测量之前进行封装,将参数写入后,效果如 \figref{07} ,进行外部连线,效果如 \figref{08} ,进行 TRANS 仿真,效果如 \figref{09} ,峰-峰值为 \(1 V\) ,交越失真明显,满足作业要求。

\qfig[07]{hw10p7.png}{模块封装}

\qfig[08]{hw10p9.png}{10 倍运放接法}

\qfig[09]{hw10p8.png}{10 倍运放效果}

\subsection{功耗测试}

静态工作点的电流如 \figref{10} ,总电流为\(760u + 3n + 3n + 106u + 5u + 200u + 5u + 10u = 1.086 mA\) 。满足条件。

\qfig[10]{hw10p10.png}{静态工作电流}
% End Here

\end{document}