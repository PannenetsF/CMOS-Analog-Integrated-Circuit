\documentclass[lang=cn,11pt,a4paper,cite=authoryear]{elegantpaper}

% 微分号
\newcommand{\dd}[1]{\mathrm{d}#1}
\newcommand{\pp}[1]{\partial{}#1}

\newcommand{\homep}[1]{\section*{Problem #1}}

% FT LT ZT
\newcommand{\ft}[1]{\mathscr{F}[#1]}
\newcommand{\fta}{\xrightarrow{\mathscr{F}}}
\newcommand{\lt}[1]{\mathscr{L}[#1]}
\newcommand{\lta}{\xrightarrow{\mathscr{L}}}
\newcommand{\zt}[1]{\mathscr{Z}[#1]}
\newcommand{\zta}{\xrightarrow{\mathscr{Z}}}

% 积分求和号

\newcommand{\dsum}{\displaystyle\sum}
\newcommand{\aint}{\int_{-\infty}^{+\infty}}

% 简易图片插入
\newcommand{\qfig}[3][nolabel]{
  \begin{figure}[!htb]
      \centering
      \includegraphics[width=0.6\textwidth]{#2}
      \caption{#3}
      \label{\chapname :#1}
  \end{figure}
}

% 表格
\renewcommand\arraystretch{1.5}


% 日期


\title{CMOS模拟集成电路原理\quad 第四周作业}
\author{范云潜 18373486}
\institute{微电子学院 184111 班}
\date{\zhtoday}

\begin{document}

\maketitle

作业内容:1. 自由确定晶体管的尺寸,通过仿真寻找NMOS和PMOS的1/f噪
声的系数KF,以及热噪声系数 \(\gamma\)

2. 对一偏置电流为100uA的五管OTA,通过晶体管的设计,使其
等效输入噪声的80\%来源于差分对M1和M2。

% \tableofcontents
\listoffigures
% Start Here

\homep{1}

测试平台:新建一个有源负载的单晶体管放大器,设计的长宽比以及使得两晶体管均工作在饱和区的电压偏置如\figref{p1f1}所示。按照需要分别将 nMOS 与 pMOS 的输入电压作为系统输入以及噪声的等效源,其特性如 \figref{p1f2} \figref{p1f3} ,可以确定我们的静态工作点,如\figref{p1f5}。并且,需要在低频下进行仿真来计算闪烁噪声。

核心公式为 

\[KF = \bar{V_i^2} WL C_{ox}^2 f\]

其中 \(V_i^2 * A_v^2 = V_o^2\)

\[\gamma = \frac{\bar{I^2}}{4 kT g_m}\]

\qfig[p1f1]{hw03p1f1.png}{基本电路设计以及偏置}

\qfig[p1f2]{hw03p1f2.png}{pMOS \(V_{out}-V_{in}\) 图像}

\qfig[p1f3]{hw03p1f3.png}{nMOS \(V_{out}-V_{in}\) 图像}

\qfig[p1f5]{hw03p1f5.png}{pMOS 静态工作点}

\qfig[p1f6]{hw03p1f6.png}{nMOS 静态工作点}



\qfig[p1f4]{hw03p1f4.png}{pMOS 噪声结果}


\qfig[p1f7]{hw03p1f7.png}{nMOS 噪声结果}

% \subhomep{\(KF\)}


根据静态工作点获得 \(g_m\)与\(C_{gs}\) ,根据 1 Hz 的噪声分析获得 \(i_d\) 与 输出电阻 \(r_x\) 

设置 pMOS 的输入为 \(V_{in}\) 并设置静态工作点为 0.8 V ,建立 AC 仿真与 NOISE 仿真,得到结果如 \figref{p1f4} ,在 1 Hz 下带入数据并换算单位得到 \(KF =4.274180e-30 C^2/cm^2\),\(\gamma = 0.784253\)


设置 nMOS 的输入为 \(V_{in}\) 并设置静态工作点为 0.65 V ,建立 AC 仿真与 NOISE 仿真,得到结果如 \figref{p1f7} ,在 1 Hz 下带入数据并换算单位得到 \(KF = 3.652783e-31 C^2/cm^2\),\(\gamma = 0.654056\)

最后,\(\gamma\) 值与教材较为吻合,\(KF_n\) 吻合较好,\(KF_p\) 吻合较差。


\lstinputlisting[caption={\(\gamma\) 的计算}]{hw03/gam.m}

\lstinputlisting[caption={\(KF\) 的计算}]{hw03/kf.m}

\homep{2}

搭建电路图如\figref{p2f1}, % p2f1 
对共模电平进行扫描来获得合适的大信号电压如\figref{p2f2}, % p2f2 
设置共模电平为 1.4 V ,
通过 op 仿真确定工作区,如\figref{p2f3}, % p2f3
全部饱和。

设置静态工作点共模电压为 1.4 V 如\figref{p2f4}% p2f4 p2f5



\qfig[p2f1]{hw03p2f1.png}{OTA 电路}

\qfig[p2f2]{hw03p2f2.png}{OTA 扫描共模电压}

\qfig[p2f3]{hw03p2f3.png}{各晶体管状态}
核心公式为

\qfig[p2f4]{hw03p2f4.png}{输出电压 DC 扫描}

\[{V_{ieq}^2 = \frac{I_{out}^2}{g_m^2}} = 2 V_{in}^2 (1 + \frac{g_{m,n}}{g_{m,p}})\]

所以为了使得来源于 pMOS 差分对的噪声占比 80\% 

\[\frac{g_{m,n}}{g_{m,p}} = 0.25\]

又 \[g_m =\sqrt{2 \mu C_{o x} \frac{W}{L} I_{D}}\]

在假定其他指标不受影响的情况下,调节 \({W}:{L}\) 。

那么近似有\[(W/L)_n:(W/L)_p = 1:16\]

设置 pMOS 宽度为 wid 便于调节宽度,如\figref{p2f5}。% p2f5

\qfig[p2f5]{hw03p2f5.png}{pMOS 宽度设为变量}

设宽度为\(220n * 16 = 3520 n\)在 1 Hz 下的噪声比例为 76.712\% ,如\figref{p2f6} 。 % p2f6
微调,设宽度为 \(3300 n\) 在 1 Hz 下噪声比例为 79.847\%,如\figref{p2f7} 。 % p2f7

\qfig[p2f6]{hw03p2f6.png}{\(W_p=3520n\) 噪声输出}

\qfig[p2f7]{hw03p2f7.png}{\(W_p=3300n\) 噪声输出}

% n 386a 131

%           p
% 580u:cox  724a
%         118.7

% 700 cox 900a
%     gm 137


% End Here

\end{document}