\documentclass[lang=cn,11pt,a4paper,cite=authoryear]{elegantpaper}

% 微分号
\newcommand{\dd}[1]{\mathrm{d}#1}
\newcommand{\pp}[1]{\partial{}#1}

\newcommand{\homep}[1]{\section*{Problem #1}}

% FT LT ZT
\newcommand{\ft}[1]{\mathscr{F}[#1]}
\newcommand{\fta}{\xrightarrow{\mathscr{F}}}
\newcommand{\lt}[1]{\mathscr{L}[#1]}
\newcommand{\lta}{\xrightarrow{\mathscr{L}}}
\newcommand{\zt}[1]{\mathscr{Z}[#1]}
\newcommand{\zta}{\xrightarrow{\mathscr{Z}}}

% 积分求和号

\newcommand{\dsum}{\displaystyle\sum}
\newcommand{\aint}{\int_{-\infty}^{+\infty}}

% 简易图片插入
\newcommand{\qfig}[3][nolabel]{
  \begin{figure}[!htb]
      \centering
      \includegraphics[width=0.6\textwidth]{#2}
      \caption{#3}
      \label{\chapname :#1}
  \end{figure}
}

% 表格
\renewcommand\arraystretch{1.5}


% 日期


\title{CMOS模拟集成电路原理\quad 第三周作业}
\author{范云潜 18373486}
\institute{微电子学院 184111 班}
\date{\zhtoday}

\begin{document}

\maketitle

作业内容:作业1:利用仿真结果,找到我们所使用工艺的 \(\mu, C_{ox}, V_{th}\)

作业2:通过公式描述 \(V_B\) 的取值范围(提示:晶体管M1-M4均需要工作在饱
和区);在Cascode电流镜中,假设有寄生电容 \(C_{par}\) ,利用公式估算并用仿真
验证该电流镜的频率特性。 (可以自由设置偏置、晶体管的尺寸以及
寄生电容)
\qfig{hw0202.png}{题目 2}


作业3
假设差分对偏置电流为200uA,W/L=20um/1um,根据理论分
析和仿真验证。
1. 计算g m >99%*g m,max 的区间。
2. 计算差分输出电流为198uA时的差分输入电压。
3. 如果需要把问题2中求得的电压扩大一倍,差分对的W需要如何修改?

% Start Here

本次作业使用 \(W/L = 1\mu m / 0.18 \mu m\)

\homep{作业1}

仿真的思路:建立电路图之后,对特定的直流点进行工作点仿真,即 OP 仿真,得到相关的直流参数如电流电压以及对应的电容等之后,通过相关公式计算栅氧的电容。

仿真,得到 nMOS \(V_{th} = 456.44 mV, C_{gs} = 1.32 fF \), 如\figref{vthn}; 得到 pMOS \(V_{th} = 319.17 mV, C_{gs} = 888.9 aF\) 如 \figref{vthp}。


\qfig[vthn]{hw0201vth.png}{nMOS 参数}

\qfig[vthp]{hw0201vthp.png}{pMOS 参数}

为了得到 \(C_{ox}\) 

\[\begin{aligned}
    C_{GS} \approx \frac{2}{3} WL C_{ox} \\ 
    % f_T = \frac{g_m}{2 \pi C_{gs}}
    C_{ox} = \frac{3}{2} \frac{C_{gs}}{WL} 
\end{aligned}\]

计算得到 \(C_{ox,n} = 0.0110 F/m^2\), \(C_{ox,p} = 0.0074 F/m^2\)

之后即可通过电流公式进行计算 \(\mu\)

\[I_d = \frac{1}{2} \mu_n C_{ox} \frac{W}{L} (V_{gs} - V_{th})^2\]

\[\mu_n = \frac{2 I_d}{C_{ox} \frac{W}{L}(V_{od})^2}\]

计算得到 \(\mu_n = 0.0240 m^2 V^{-1}s^{-1}\) , \(\mu_p = 1.3934e-04V^{-1}s^{-1} \)


\homep{作业 2}

仿真的思路是建立 \figref{fuckcir},需要在偏置电流的基础上对各个点的电压进行预先设计,据此确定晶体管尺寸。建立之后需要先进行直流 OP 测试,确保各个晶体管工作在正常的饱和区,并且电流镜可以正常工作。进行小信号测试时依靠 AC 测试对电流源的驮载小信号进行扫频计算,通过电压-电流增益确定特征频率。

\subhomep{1}

对 1,3 管列式,记 \(V_x\) 为 M3 漏极电压,\(V_y\) 为 M3 源极电压:

\[\begin{aligned}
    V_B - V_x &< V_{th} \\
    V_B - V_y &> V_{th} \\ 
    V_x - V_y &< V_{th} \\ 
    V_x &> V_{th}
\end{aligned}\]

那么 \[V_x < V_{th} + V_y < V_B < V_{th} + V_x < V_y + 2 V_{th}\]

% 以 \(V_th = 0.45 V\) 为基准,设置 \(V_x = 1.5 V\), \((W/L)_1 = 20\) ,偏置电流为 \(I_d = 3 mA\),设\(V_y = 1.2 V\), \(V_B  = 2.2 V\),得到 \((W/L)_3 = 75\)以此为估计进行仿真。

% r3 6 k r1 1.5k
\subhomep{2}

设 \(V_x = 1 V\) 以 \(V_{th} = 0.5 V\) 预估,可以使\(V_y  =0.7 V\) \(V_B = 1.4 V\) ,设置偏置电流为 \(0.1 mA\) 。据此计算得到:\((W/L)_3 = 7.2\mu / 0.18 \mu\) \((W/L)_1 = 0.56 
\mu/0.18\mu \)

最终搭建电路如  \figref{fuckcir},频率响应如 \figref{fuckfre},特征频率为 \(30 M Hz\)


\qfig[fuckcir]{hw0202circuit.png}{Cascode 电路}


\qfig[fuckfre]{hw0202polarpoint.png}{Cascode 电路频响}

手工计算:

\[r_o \approx \frac{1}{\lambda I_D} \approx = \frac{1}{0.1 \cdot 0.1 / 1000} = 1e6 \Omega\]

那么特征频率为

\[f_t = \frac{1}{2 \pi C r_{eq}} = \frac{1}{2 \pi \cdot 100 f \cdot 50 k} = 31830988.61838 Hz = 31.8 MHz\] 

\homep{作业 3} 

仿真的思路是建立一个基本差动对,通过理想电流源驱动差分对,锁定电流。对差动对的输入使用等大反向的驱动,通过 DC 扫描对输入电压进行驱动,并通过微分操作计算跨导。

\subhomep{1}

搭建电路,如\figref{damn} ,对 \(V_{in}\) 扫描,结果如\figref{damngm}。 \(g_m = 1.2 m\) ,得到 \(0.99 g_m = 1.188 m\) ,图中结果为 \(V_{in} = 2 *16 = 32 mV\),如\figref{damnvin}。


手工计算 

\[g_m=\frac{\partial \Delta I_{D}}{\partial \Delta V_{i n}}=\frac{1}{2} \mu_{n} C_{o x} \frac{W}{L} \frac{\frac{4 I_{S S}}{\mu_{n} C_{o x} W / L}-2\left(\Delta V_{i n}\right)^{2}}{\sqrt{\frac{4 I_{S S}}{\mu_{n} C_{o x} W / L}-\left(\Delta V_{i n}\right)^{2}}}\]

仿真得到 \(\mu_n C_{ox} = 2 I_d / (V_{od}^2) / (W/L) = 0.00039157 = k\) ,带入得到

\[g_{m,max} \sqrt{I_{ss} \mu_nC_{ox} W/L}  = 0.0017699\]

% \[\]

\subhomep{2}

\(I = 198 u\) 时, \(\Delta V_{in} = 2 * 0.14 = 0.28 V\)。如\figref{198}

手工计算得到

\[V_{in} =\sqrt{2 I_{S S} /\left(\mu_{n} C_{o x} W / L\right)} = 0.15981 V\] 

那么 \(\Delta V_{in} = 0.32 V\)

\subhomep{3}

此时可以认为,几乎是 \(\Delta V_{in}\) 为最大值的时候,将其扩大为 2 倍,那么根据公式 $\Delta V_{\text {in}}=\sqrt{2 I_{S S} /\left(\mu_{n} C_{o x} W / L\right)}$ 得到, \(W/L\) 变为 1/4 倍,即 \(W = 10 \mu m\)。

分别仿真出的电压是 \(0.15V\) \(0.26V\) 如\figref{vout1} \figref{vout2}


\qfig[damn]{hw0203circuit.png}{差动电路}

\qfig[damngm]{hw0203ans.png}{差动电路结果}

\qfig[damnvin]{hw0203gm.png}{差动电路区间}

% \qfig[damnvin]{hw0203vin.png}{区间}

\qfig[damnvins]{hw0203vins.png}{差分电压}


\qfig[198]{hw0203sub2.png}{198 uA 对应电压}


\qfig[vout1]{hw0203orin.png}{抽取电压}

\qfig[vout2]{hw0203orin2.png}{抽取电压}

% End Here


\end{document}