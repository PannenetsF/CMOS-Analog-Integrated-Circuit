\documentclass[lang=cn,11pt,a4paper,cite=authoryear]{elegantpaper}

% 微分号
\newcommand{\dd}[1]{\mathrm{d}#1}
\newcommand{\pp}[1]{\partial{}#1}

\newcommand{\homep}[1]{\section*{Problem #1}}

% FT LT ZT
\newcommand{\ft}[1]{\mathscr{F}[#1]}
\newcommand{\fta}{\xrightarrow{\mathscr{F}}}
\newcommand{\lt}[1]{\mathscr{L}[#1]}
\newcommand{\lta}{\xrightarrow{\mathscr{L}}}
\newcommand{\zt}[1]{\mathscr{Z}[#1]}
\newcommand{\zta}{\xrightarrow{\mathscr{Z}}}

% 积分求和号

\newcommand{\dsum}{\displaystyle\sum}
\newcommand{\aint}{\int_{-\infty}^{+\infty}}

% 简易图片插入
\newcommand{\qfig}[3][nolabel]{
  \begin{figure}[!htb]
      \centering
      \includegraphics[width=0.6\textwidth]{#2}
      \caption{#3}
      \label{\chapname :#1}
  \end{figure}
}

% 表格
\renewcommand\arraystretch{1.5}


% 日期


\title{CMOS模拟集成电路原理\quad 第十一周作业}
\author{范云潜 18373486}
\institute{微电子学院 184111 班}
\date{\zhtoday}

\begin{document}

\maketitle

% 作业内容:

\tableofcontents

\listoffigures

\section{流程简述}

SAR-ADC (Successive-approximation Register ADC) 是利用 DAC 模块逐次二分逼近最终得到合适的 ADC 值的一种方法。接下来简要介绍整体流程。

\begin{enumerate}
    \item 在一次转换的开始,如 \figref{01} ,对输入信号进行采样,所有由 DAC 输出控制的开关摆向底板,底板连接到输入信号,采样端此时需要接地,那么电容两端的电势差为 \(0 - V_{in}\) 。
    \item 采样完成后,需要将得到的信号进行保持,如 \figref{02} 将底板接到参考电压,
\end{enumerate}

\qfig[01]{hw11p1.png}{对输入信号进行采样}

\qfig[02]{hw11p2.png}{对输入信号进行保持}

\qfig[03]{hw11p3.png}{逼近信号}

\section{逼近器设计}




% Start Here

% End Here

\end{document}