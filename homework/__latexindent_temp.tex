\documentclass[lang=cn,11pt,a4paper,cite=authoryear]{elegantpaper}

% 微分号
\newcommand{\dd}[1]{\mathrm{d}#1}
\newcommand{\pp}[1]{\partial{}#1}

\newcommand{\homep}[1]{\section*{Problem #1}}

% FT LT ZT
\newcommand{\ft}[1]{\mathscr{F}[#1]}
\newcommand{\fta}{\xrightarrow{\mathscr{F}}}
\newcommand{\lt}[1]{\mathscr{L}[#1]}
\newcommand{\lta}{\xrightarrow{\mathscr{L}}}
\newcommand{\zt}[1]{\mathscr{Z}[#1]}
\newcommand{\zta}{\xrightarrow{\mathscr{Z}}}

% 积分求和号

\newcommand{\dsum}{\displaystyle\sum}
\newcommand{\aint}{\int_{-\infty}^{+\infty}}

% 简易图片插入
\newcommand{\qfig}[3][nolabel]{
  \begin{figure}[!htb]
      \centering
      \includegraphics[width=0.6\textwidth]{#2}
      \caption{#3}
      \label{\chapname :#1}
  \end{figure}
}

% 表格
\renewcommand\arraystretch{1.5}


% 日期


\title{CMOS模拟集成电路原理\quad 第十周作业}
\author{范云潜 18373486}
\institute{微电子学院 184111 班}
\date{\zhtoday}

\begin{document}

\maketitle


\tableofcontents

% Start Here

\section{电路理解}

本次实验电路如 \figref{01} ,进行以下几个模块的拆分:

\begin{itemize}
    \item 差分对: MN1+MI(1,2) 。输出差分电流
    \item 电流放大器: M(5-8, 11-14) 。 对输入的差分电流转化成电压输出。将 \(I_{+} - I_{-}\) 变化为 \(I_{in}\) 。MA(5-10) ,可看作电阻,为 MA(3,4) 提供偏置
    \item 输出级: 通过 MA(1,2) 放大电压信号。
    %TODO: MN3 是个 NMOS? 
\end{itemize}

分析需要的放大倍数: 

\[\frac{10 A}{A + 11} = 9.5\]

解得 \(A = 209\) ,五倍冗余 \(A = 1045\) 。

这个三级的放大,第一级的放大是 \(I/V = g_{m,I}\) ,第二级是 \(\)

\qfig[01]{hw10p1.png}{线性跨导回路电路}

% End Here

\end{document}