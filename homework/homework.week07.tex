\documentclass[lang=cn,11pt,a4paper,cite=authoryear]{elegantpaper}

% 微分号
\newcommand{\dd}[1]{\mathrm{d}#1}
\newcommand{\pp}[1]{\partial{}#1}

\newcommand{\homep}[1]{\section*{Problem #1}}

% FT LT ZT
\newcommand{\ft}[1]{\mathscr{F}[#1]}
\newcommand{\fta}{\xrightarrow{\mathscr{F}}}
\newcommand{\lt}[1]{\mathscr{L}[#1]}
\newcommand{\lta}{\xrightarrow{\mathscr{L}}}
\newcommand{\zt}[1]{\mathscr{Z}[#1]}
\newcommand{\zta}{\xrightarrow{\mathscr{Z}}}

% 积分求和号

\newcommand{\dsum}{\displaystyle\sum}
\newcommand{\aint}{\int_{-\infty}^{+\infty}}

% 简易图片插入
\newcommand{\qfig}[3][nolabel]{
  \begin{figure}[!htb]
      \centering
      \includegraphics[width=0.6\textwidth]{#2}
      \caption{#3}
      \label{\chapname :#1}
  \end{figure}
}

% 表格
\renewcommand\arraystretch{1.5}


% 日期


\title{CMOS模拟集成电路原理\quad 第一周作业}
\author{范云潜 18373486}
\institute{微电子学院 184111 班}
\date{\zhtoday}

\begin{document}

\maketitle

作业内容:❑ 已知要求GBW DM =50MHz,GBW CM =100MHz, C L =5pF。设计一共
模\&差模相位裕度均大于70的运放。通过仿真给出:
▪ 差模增益
▪ 功耗
▪ 共模抑制比CMRR

将上述设计的差分运放,通过电阻设置成10倍放大,观察输入差模和共
模信号分别有100mVpp,10kHz的正弦信号时,差模输出信号的大小,
并分析是否符合预期。

\qfig[00]{hw07p1.png}{题目图}

\listoffigures
% \tableofcontents


\section{分析电路}

M1 与 M2 构成一个 Cascode ,输出到一个共栅极,通过一个源随器和电阻对消除差模,之后通过对电流源的电流吸取,途径 M7 反馈到输出端。那么 \(GBW_{DM} = g_{m1}/2 \pi C_L\) ,\(GBW_{CM} = g_{m7} / 2 \pi 2 C_L\) 。

首先是差模放大部分,其跨导来自于 M1 ,因此为了使得 \(GBW_{DM}\) 大, 应该调大其 \(g_M\) ;同时注意到, 共栅极的 M4 是一个复制管, 和误差放大器的电流应该保持一致;最后是共模反馈部分, 其跨导来自 M7 。综上所述,为了增益带宽积大, 需要调大 M1 和 M7 的电流,为了使得裕度更高, 次级点的电流也应更大。同时 M4 的多次出现使得共栅极放大器的电流已经确定,因此需要控制的电流仅有源随器一级缓冲。

\section{整体电路搭建}

整体电路如 \figref{02} ,接下来主要通过 \(W = 2  I \cdot L / k  / V_{gst}^2\) 进行估算,而阈值会产生一定的变化,通过打表进行估算,如 \tabref{t1} 。


\qfig[02]{hw07p2.pdf}{电路结构}

\begin{table}[htb]
    \caption{在 1 微米下的阈值}
    \centering
    \label{t1}
    \begin{tabular}{cccc}
    \toprule
    n                       & 阈值                      & p                    & 阈值                   \\
    \midrule
    0                       & 437.8                   & 1.8                  & 443.6                \\
    0.1                     & 466.7                   & 1.7                  & 474                  \\
    0.2                     & 494.1                   & 1.6                  & 503.4                \\
    0.3                     & 520.4                   & 1.5                  & 531.3                \\
    0.4                     & 545                     & 1.3                  & 531.3                \\
    0.5                     & 569.9                   &                      &                      \\
    \multicolumn{1}{l}{0.6} & \multicolumn{1}{l}{593} & \multicolumn{1}{l}{} & \multicolumn{1}{l}{} \\
    \multicolumn{1}{l}{0.7} & \multicolumn{1}{l}{615} & \multicolumn{1}{l}{} & \multicolumn{1}{l}{} \\
    \multicolumn{1}{l}{0.8} & \multicolumn{1}{l}{639} & \multicolumn{1}{l}{} & \multicolumn{1}{l}{} \\
    \multicolumn{1}{l}{0.9} & \multicolumn{1}{l}{657} & \multicolumn{1}{l}{} & \multicolumn{1}{l}{} \\
    \multicolumn{1}{l}{1.0} & \multicolumn{1}{l}{657} & \multicolumn{1}{l}{} & \multicolumn{1}{l}{} \\ 
    \bottomrule
    \end{tabular}
\end{table}

计算得到两个关键管子 M1 和 M7 的跨导,分别是 \(1.5 m\) , \(6 m\) ,进而可以得到两路的电流,同时 M4 的复制导致共源共栅极的电流也确定了,自行设定源随器一级的电流为 \(20 u\) 。最终全部饱和,如 \figref{03} 。

\qfig[03]{hw07p3.png}{饱和状态}

\section{性能}

{完成共模反馈后的差分增益} 如 \figref{04} 。

\qfig[04]{hw07p4.png}{差模增益}

共模反馈的增益,保证共模状态的同时,开启反馈的小信号,如 \figref{05} ,\figref{005} 。断开整体的差分电路,用理想电压源承载对应的静态工作点,测量 输出和误差放大输入的比。

\qfig[05]{hw07p5.png}{共模反馈电路}

\qfig[005]{hw07p005.png}{共模反馈增益}

源随器级的电流可以用 OP 进行求解,如 \figref{06} ,是 \(170 \mu\) 。总功耗为 \(2 \times (170 \mu + 1.2 m + 0.2 m \time 2 + 0.8m \times 4) 1.8 = 6.066 mW\) 。

\qfig[06]{hw07p6.png}{源随器级电流}

测量 CMRR 实用电阻匹配法,电路如 \figref{07} ,测量得到共模增益 如 \figref{08} ,\(CMRR = A_{dm}/A_{cm} = 5.5 / 5.4 m = 100 \)。

\qfig[07]{hw07p7.png}{共模电路}


\qfig[08]{hw07p006.png}{共模增益}

对其共模加入正弦信号,电路如 \figref{004} ,共模波形如 \figref{003} 。


\qfig[004]{hw07p004.png}{共模输入正弦波电路}


\qfig[003]{hw07p003.png}{共模输入正弦波波形}

对其差模加入正弦信号,电路如 \figref{004} ,差模波形如 \figref{003} 。


\qfig[002]{hw07p002.png}{差模输入正弦波电路}


\qfig[001]{hw07p001.png}{差模输入正弦波波形}

% 001 差模放大 002 电路
% 
% \section{差模电路}

% 首先搭建差模电路,如 \figref{02} 。

% 根据估算, \(g_{M1} = 1.5 m\) ,设置偏置电流。为了使得共栅极放大器工作正常,设置节点 1 的静态电压为 \(0.5 V\) 。为了承载大电流,过驱动电压需要调高,仿真得到 M2 的阈值为 \(0.45 V\) ,设其栅电压为 \(0.65 V\) ; M1 阈值为 \(0.65 V\) ,设其栅电压为 \(0.3 V\) 。求解两个管子的宽长比。


% 第二级的电流设置为和第一级几乎吻合,设置 M3 M4 栅电压为 \(1.2 V 1.15 V\) ,求解宽长比。

% 最终结果如 \figref{03}

% \qfig[03]{hw07p3.png}{差分电路参数}

% 进行交流仿真,观察其增益波形,如 \figref{04}。其增益带宽积和裕度满足要求。

% \qfig[04]{hw07p4.png}{差分增益}

% \section{共模电路}

% 将差模电路测试参数写入各个管子,封装为元件,端口如 \figref{05} ,连接共模反馈电路,如 \figref{06} 。共模电路的输入在误差放大器上,是来自于经过电阻消除差模信号后的差分电路输出,通过误差放大器漏极,最终反馈到我们的输出端,关闭其他交流信号,切断回路,在误差放大器加上激励,如 \figref{08}。参数如 \figref{13}

% \qfig[05]{hw07p5.png}{差分电路端口图}

% \qfig[06]{hw07p6.png}{共模反馈电路}

% \qfig[08]{hw07p8.png}{回路切断}

% \qfig[13]{hw07p13.png}{共模电路参数}
% 基于以上的理解,进行对应的 AC 仿真,仿真结果如 \figref{07} ,满足要求。

% \qfig[07]{hw07p7.png}{共模增益}

% \section{性能分析}

% 差模增益,在 AC 仿真中设置差分信号 \lstinline{vdif} 的交流振幅为 1 ,增益表达式为 \lstinline{vout/vin} ,仿真结果如 \figref{09} 。

% \qfig[09]{hw07p9.png}{差分增益}

% 功耗,在 OP 仿真可以得到非电流偏置的两条支路的电流为 \(100 \mu A\) 如 \figref{10},两个电流源分别是 \(400\mu A\) , \(600 \mu A\) ,总功耗为 \((400 + 600 + 100 \cdot 2) 1.8 \mu W = 2.16 mW\) 。

% \qfig[10]{hw07p10.png}{共模反馈电流}

% 为了更方便的进行 CMRR 仿真,进一步封装模块,如 \figref{11} 。为了进一步的分析,此处的连线是为了使用电阻匹配法,电压仅有共模模式, 如\figref{12}。结果如 \figref{14} 。

% \qfig[11]{hw07p11.png}{整体封装}

% \qfig[12]{hw07p12.png}{CMRR 连线}

% \qfig[14]{hw07p14.png}{CMRR 结果}

% \section{TRANS 仿真}

% 正弦源作为共模输入,电路如 \figref{15} ,效果如 \figref{16} ,不失真。

% \qfig[15]{hw07p15.png}{共模正弦输入电路}

% \qfig[16]{hw07p16.png}{共模正弦输入输出电压对比}

% 正弦源作为差模输入,电路如 \figref{17} ,效果如 \figref{18} ,不失真。

% \qfig[17]{hw07p17.png}{差模正弦输入电路}

% \qfig[18]{hw07p18.png}{差模正弦输入输出电压对比}

% \section*{后记}

% 关于电流镜的问题,我很抱歉,但是事实是我尽力了。在过去的一周里,我完成了六门作业,共用 27 h 31 min ,其中模电用时 19 h 36 min 。经过的不断尝试,电流镜的替换破坏了我的结构,我确信我没有精力再去完成一次这样的设计。
% % Start Here

% End Here

\end{document}