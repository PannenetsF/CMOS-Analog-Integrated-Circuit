\ifx\mainclass\undefined
\documentclass[cn,11pt,chinese,black,simple]{../elegantbook}
% 微分号
\newcommand{\dd}[1]{\mathrm{d}#1}
\newcommand{\pp}[1]{\partial{}#1}

\newcommand{\homep}[1]{\section*{Problem #1}}

% FT LT ZT
\newcommand{\ft}[1]{\mathscr{F}[#1]}
\newcommand{\fta}{\xrightarrow{\mathscr{F}}}
\newcommand{\lt}[1]{\mathscr{L}[#1]}
\newcommand{\lta}{\xrightarrow{\mathscr{L}}}
\newcommand{\zt}[1]{\mathscr{Z}[#1]}
\newcommand{\zta}{\xrightarrow{\mathscr{Z}}}

% 积分求和号

\newcommand{\dsum}{\displaystyle\sum}
\newcommand{\aint}{\int_{-\infty}^{+\infty}}

% 简易图片插入
\newcommand{\qfig}[3][nolabel]{
  \begin{figure}[!htb]
      \centering
      \includegraphics[width=0.6\textwidth]{#2}
      \caption{#3}
      \label{\chapname :#1}
  \end{figure}
}

% 表格
\renewcommand\arraystretch{1.5}


% 日期

\begin{document}
\fi 
\def\chapname{07r2r}

% Start Here
\chapter{轨到轨}

\section{互补差分对结构}

是解决轨到轨输入的基本思路:\(V_{INCM,N} > 1.1 V, V_{INCM, P} < V_{DD} - 1.1 V\) 在 \(V_{DD} > 2.2 V \) 可实现。当共模输入接近电源上下限的时候,只有一组差分对导通,变化的跨导会导致大量的非线性失真。

因此,我们需要提升两个端点的跨导。

\section{跨导平衡技术}

\subsection{三倍电流镜}

\(g_{mn} + g_{mp} = ct1 \rightarrow \sqrt{I_{BN}} + \sqrt{I_{BP}} = ct3\)

如何自适应产生偏置电流:在不工作时,将另一侧电流增加到 3 倍保证其根号不变。最终波动为 \(\Delta g_m / g_m = 15 \%\) 。

\subsection{稳压二极管/奇纳二极管}

在边缘处,稳压管停止。\(25 \%\)

为了加快电压的变化,使用运放和源随器加快电压变化。 \(6 \%\)

\subsection{亚反型区}

在亚反型区中 \(g_m = \dfrac{I_{D_wi}}{nkT/q}\) ,但是 \(n\) 较难控制。

\[I_{BN} + \frac{n_n}{n_p} I_{BP} = ct\]

\subsection{反馈调制}

副本调制。 \(4\%\) 


\section{超低电压的轨到轨}

加入电压偏移电路。

% End Here

\let\chapname\undefined
\ifx\mainclass\undefined
\end{document}
\fi 