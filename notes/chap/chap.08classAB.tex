\ifx\mainclass\undefined
\documentclass[cn,11pt,chinese,black,simple]{../elegantbook}
% 微分号
\newcommand{\dd}[1]{\mathrm{d}#1}
\newcommand{\pp}[1]{\partial{}#1}

\newcommand{\homep}[1]{\section*{Problem #1}}

% FT LT ZT
\newcommand{\ft}[1]{\mathscr{F}[#1]}
\newcommand{\fta}{\xrightarrow{\mathscr{F}}}
\newcommand{\lt}[1]{\mathscr{L}[#1]}
\newcommand{\lta}{\xrightarrow{\mathscr{L}}}
\newcommand{\zt}[1]{\mathscr{Z}[#1]}
\newcommand{\zta}{\xrightarrow{\mathscr{Z}}}

% 积分求和号

\newcommand{\dsum}{\displaystyle\sum}
\newcommand{\aint}{\int_{-\infty}^{+\infty}}

% 简易图片插入
\newcommand{\qfig}[3][nolabel]{
  \begin{figure}[!htb]
      \centering
      \includegraphics[width=0.6\textwidth]{#2}
      \caption{#3}
      \label{\chapname :#1}
  \end{figure}
}

% 表格
\renewcommand\arraystretch{1.5}


% 日期

\begin{document}
\fi 
\def\chapname{08classAB}

% Start Here
\chapter{AB 类放大器}

为什么需要推挽输出?输出级只能输出电流,而不能吸收电流。

\section{AB 类}

A 类:偏置电流大于摆幅;B 类:无偏置电流;AB 类:偏置电流小于摆幅。

AB 类是功耗和失真的权衡。

要求轨到轨输出,并精确控制静态电流(低功耗,独立于电源电压),大电流驱动能力,面积尽量小。

驱动能力由 \(I_{out}/I_Q\) 定义, AB 类的输出需要比线性增长更快,而差分对是一个限制因素。

\section{电流控制}

\subsection{交叉耦合差分对}

在无差分时,与未交叉的静态工作点一致;出现差分时,作用在两个管子 \(V_{GS}\) 上。

\subsection{动态偏置}

\subsection{线性跨导}






% End Here

\let\chapname\undefined
\ifx\mainclass\undefined
\end{document}
\fi 