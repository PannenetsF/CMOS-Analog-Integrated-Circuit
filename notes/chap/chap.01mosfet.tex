\ifx\mainclass\undefined
\documentclass[cn,11pt,chinese,black,simple]{../elegantbook}
% 微分号
\newcommand{\dd}[1]{\mathrm{d}#1}
\newcommand{\pp}[1]{\partial{}#1}

\newcommand{\homep}[1]{\section*{Problem #1}}

% FT LT ZT
\newcommand{\ft}[1]{\mathscr{F}[#1]}
\newcommand{\fta}{\xrightarrow{\mathscr{F}}}
\newcommand{\lt}[1]{\mathscr{L}[#1]}
\newcommand{\lta}{\xrightarrow{\mathscr{L}}}
\newcommand{\zt}[1]{\mathscr{Z}[#1]}
\newcommand{\zta}{\xrightarrow{\mathscr{Z}}}

% 积分求和号

\newcommand{\dsum}{\displaystyle\sum}
\newcommand{\aint}{\int_{-\infty}^{+\infty}}

% 简易图片插入
\newcommand{\qfig}[3][nolabel]{
  \begin{figure}[!htb]
      \centering
      \includegraphics[width=0.6\textwidth]{#2}
      \caption{#3}
      \label{\chapname :#1}
  \end{figure}
}

% 表格
\renewcommand\arraystretch{1.5}


% 日期

\begin{document}
\fi 
\def\chapname{01mosfet}

% Start Here
\chapter{MOSFET}

\section*{回顾}

MOS 的晶体电流公式:

线性区时: 

\[
I_d = \mu_0 C_{ox} \frac{W}{L} (V_{gs} - V_{th} - \frac{1}{2} V_{ds}) V_{ds} 
\]

饱和区时:

\[
I_d = \frac{1}{2} \mu_0 C_{ox} \frac{W}{L} (V_{gs} - V_{th})^2
\]

\section{线性区:电阻}

在线性区满足 \(V_{DS} < V_{GS} - V_{th}\),当沟道打开时,沟道高度与 \(V_{gs}\) 成比例,通过面积法理解电流中的 \(V_{ds}/2 \) 来源。

小信号的导通电阻求解为:

\[
R = \frac{\pp{V}}{\pp{I}} \approx \frac{1}{\mu_0 C_{ox} \dfrac{W}{L} (V_{gs} - V_{th}) }  
\]


电子迁移率大概是 \(\mu_n \approx 600 \text{cm}^2/Vs\), \(\mu_p \approx 250 \text{cm}^2/Vs\) 。 栅氧层电容 \(C_{ox} = \frac{\epsilon_{ox}}{t_{ox}}\) ,基本可以按照特征尺寸 \(L_{min}\) 估计栅氧厚度 \(t_{ox} = \dfrac{L_{min}}{50}\) 。一般使用 \(\text{cm}^2\) 相关的单位。定义工艺量 \(KP_{n} = \mu_n C_{ox,n}\)。

根据以上的知识可以对 MOSFET 的电阻进行快速的估算。工艺越小,由于特征尺寸小,电阻更小; PMOS 由于迁移率小,电阻更大。

\section{饱和区:放大器}

在 \(V_{ds} > V_{gs} - V_{th}\) 时,进入饱和区。对于\(I_d\)公式,如何理解其系数 \(1/2\) 以及平方项:


\[
I_d = \frac{1}{2} \mu_0 C_{ox} \frac{W}{L} (V_{gs} - V_{th})^2
\]

同样是通过沟道图理解, \(V_{ds}\) 最多造成 \(V_{gs} - V_{th}\) 的影响,沟道越长,调制效应越小。

其跨导定义如下,最后一种形式最常见,需要记忆:和 \(I_d\) 直接相关,商为能量效率。

\[
\begin{aligned}
    g_m = \frac{\pp{I_d}}{\pp{V_{gs}}} &= \mu_0 C_{ox} \frac{W}{L}(V_{gs} - V_{th}) \\
    &= \sqrt{{2 I_d \cdot \mu_0 C_{ox} \frac{W}{L}}\\
    &= \frac{2 I_d}{V_{gs} - V_{th}}
\end{aligned}
\]

可以看到,跨导和漏极电流有着密切的要求,对于 \(g_m \propto \sqrt{I_d}\) 在测试中尺寸是固定的,对于 \(g_m \propto I_d\) 在设计中偏置固定。

其输出电阻 \(r_0 = V_{ds} / I_d \approx 1 / (\lambda I_d)\) 。\(\lambda = 1/(V_E \cdot L)\) , \(V_E\) 是工艺相关的量,\(L\) 是沟道长度。一般来说,\(V_{E,n} = 4 \text{V/}\mu mL\),\(L = 1 \mu m\)

\subsection{单晶体管放大器}

对于共源放大器:

\[A = g_m r_0 = \frac{2 I_d}{V_{gs} - V_{th}} \cdot \frac{V_E L}{I_D} = \frac{2 V_E L}{V_{gs} - V_{th}}\]

其中 \(V_{gs} - V_{th} \approx 0.2 V\)

运算放大器的设计存在 Trade-off 如 \tabref{tab:01:1}:对于 \(g_m \approx \frac{2 I_{ds}}{V_{gs} - V_{th}}\) ,\(A = \dfrac{2 V_E L}{V_{gs} - V_{th}}\),跨导越大,速度越快。

\begin{table}[htb]
    \centering
    \caption{Trade-off}\label{tab:01:1}
    \begin{tabular}{lll}
        \hline
        & 高增益  & 高速   \\ \hline
\(V_{gs}-V_{th}\) & down & up   \\
\(L\)             & up   & down \\ \hline
    \end{tabular}
\end{table}

\section{弱反型区}

对于弱反型区有,其中 \(n > 1\) 

\[I_{d,wi} = I_{d0} \frac{W}{L} e^{\frac{V_{gs}}{nkT/q}}\]

\[g_{m,wi} = I_{d,wi} \frac{1}{n k T / q}\]

场效应管是一个水平的 BJT 三极管,漏极的反向偏置相册二极管,栅级电压的增加会降低二极管的电势壁垒,主要电流为扩散电流而不是漂移电流,导电的通道中电势几乎不变,但是离子浓度成线性变化。

\[I_{d,wi} = I_{d0} \frac{W}{L} e^{\frac{V_{gs}}{nU_T}} \cdot \left[1 - exp(-\frac{V_{ds}}{U_T})\right]\] 

对 \(V_{ds} > 4 U_T\) 称为饱和,最后一项可以忽略。利用强反型区和弱反型区跨导相等时的电压条件,可以计算其交界点,约为\(V_{gs}-V_{th} = 2 n k T / q \approx 70 mV \),这是和工艺独立的。此时电流约为零点几个微安。保证 \(V_{gs} - V_{th} > 0.2 V\) 可以保证不同工艺中均在强反型区工作。

对于强反型区能量效率为 \(2 / (V_{gs}-V_{th})\),一般为20,弱反型区的效率几乎是一个常数。

\section{弱反型区以及强反型区:EKV 模型}

% End Here

\let\chapname\undefined
\ifx\mainclass\undefined
\end{document}
\fi 