\ifx\mainclass\undefined
\documentclass[cn,11pt,chinese,black,simple]{../elegantbook}
% 微分号
\newcommand{\dd}[1]{\mathrm{d}#1}
\newcommand{\pp}[1]{\partial{}#1}

\newcommand{\homep}[1]{\section*{Problem #1}}

% FT LT ZT
\newcommand{\ft}[1]{\mathscr{F}[#1]}
\newcommand{\fta}{\xrightarrow{\mathscr{F}}}
\newcommand{\lt}[1]{\mathscr{L}[#1]}
\newcommand{\lta}{\xrightarrow{\mathscr{L}}}
\newcommand{\zt}[1]{\mathscr{Z}[#1]}
\newcommand{\zta}{\xrightarrow{\mathscr{Z}}}

% 积分求和号

\newcommand{\dsum}{\displaystyle\sum}
\newcommand{\aint}{\int_{-\infty}^{+\infty}}

% 简易图片插入
\newcommand{\qfig}[3][nolabel]{
  \begin{figure}[!htb]
      \centering
      \includegraphics[width=0.6\textwidth]{#2}
      \caption{#3}
      \label{\chapname :#1}
  \end{figure}
}

% 表格
\renewcommand\arraystretch{1.5}


% 日期

\begin{document}
\fi 
\def\chapname{03noise}

% Start Here
\chapter{噪声}

为什么噪声很重要?是模拟电路中第一梯队的参数,需要借此衡量系统的精度以及稳定性。定义信噪比(Signal-to-Noise Ratio)为 

\[SNR = \frac{P_{sig}}{P_{noise}}, \text{ where } P_{sig} \propto V_{DD}^2, P_{noise} \propto kT/V\]

工艺的进化使得 \(SNR\) 不断下降,为了维持 \(SNR\) 那么电容也会增加,功耗就会提升。低功耗设计需要对噪声的深入理解。

\section{噪声的类型}

\begin{itemize}
    \item 人为的干扰
    \item 信号耦合,如电容、电感、基底以及键合线(wire bounding,电感纳亨级)
    \item 电源噪声
    \item 以上可以通过差分电路以及版图技巧来解决。
    \item 器件噪声:电荷的不连续性造成,是根本性的噪声
\end{itemize}

\section{噪声的表现形式}

时域上随机出现,通过概率表征,一般均值为 0 ,可以时域功率进行分析。在频域上在低频区出现闪烁噪声(Flicker Noise),高频区出现白噪声,使用功率谱密度分析 (\(V^2/Hz\))。
通过积分获得一定带宽内总的噪声。

噪声因子\footnote{Noise Figure : \(NF = 10 \log F\)},定义系统的内噪声为 \(N_a\),增益为 \(G\),源的噪声为 \(N_i\):

\[F = \frac{P_{noise, out}}{P_{noise, source}} = \frac{N_o}{N_i G} = \frac{N_i G + N_a}{N_i G}\]

由于表现形式带有增益项,一般使用输入参考噪声进行度量。

\[F = \frac{\overline{i_s}^2 + |\overline{e_n}^2 + \overline{i_n}^2|}{\overline{i_s}^2} = \frac{SNR_{out}}{SNR_{in}}\]

实质上就是输出端和输入端的信噪比之比。

\[F_n = F_1 + \frac{F_2 - 1}{G_1} + \cdots + \frac{F_n - 1}{G_1 G_2 \cdots G_n}\]

\section{电阻热噪声}

功率谱

\[{\dd{}\overline{v_n^2} = 4 k T R}, \overline{\dd{i_n^2}} = 4kT/R\]

对常温下 1 k电阻,\(40 nV/\sqrt{Hz}\)
\section{晶体管噪声}

饱和区的沟道电阻是和其他电阻一样的热噪声,一般 \(\gamma \approx 2/3, 1, 4/3\)分别对应 0.35, 0.18, 0.13 \(nm\)。

\[\dd{\overline{i_n^2}} = 4 k T / R = 4 k T \gamma g_m\]

栅级的多晶硅也存在一个电阻 \(R_G\) 。

闪烁噪声 

\[\dd{\overline{v_{i,eq,f}^2}} = \frac{KF_F}{WLC_{ox}^2}\frac{\dd{f}}{f}\]

pMOS 噪声小,nMOS 几乎是 pMOS 的 40 倍。

转角频率:偏置电流越大,白噪声越低,拐角频率就越大。

\section{等效噪声}



\section{}
% End Here

\let\chapname\undefined
\ifx\mainclass\undefined
\end{document}
\fi 