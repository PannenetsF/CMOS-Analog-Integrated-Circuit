\ifx\mainclass\undefined
\documentclass[cn,11pt,chinese,black,simple]{../elegantbook}
% 微分号
\newcommand{\dd}[1]{\mathrm{d}#1}
\newcommand{\pp}[1]{\partial{}#1}

\newcommand{\homep}[1]{\section*{Problem #1}}

% FT LT ZT
\newcommand{\ft}[1]{\mathscr{F}[#1]}
\newcommand{\fta}{\xrightarrow{\mathscr{F}}}
\newcommand{\lt}[1]{\mathscr{L}[#1]}
\newcommand{\lta}{\xrightarrow{\mathscr{L}}}
\newcommand{\zt}[1]{\mathscr{Z}[#1]}
\newcommand{\zta}{\xrightarrow{\mathscr{Z}}}

% 积分求和号

\newcommand{\dsum}{\displaystyle\sum}
\newcommand{\aint}{\int_{-\infty}^{+\infty}}

% 简易图片插入
\newcommand{\qfig}[3][nolabel]{
  \begin{figure}[!htb]
      \centering
      \includegraphics[width=0.6\textwidth]{#2}
      \caption{#3}
      \label{\chapname :#1}
  \end{figure}
}

% 表格
\renewcommand\arraystretch{1.5}


% 日期

\begin{document}
\fi 
\def\chapname{04cmrr}

% Start Here
\chapter{失调与 CMRR}

\section{失调}

失调电压是输出电压为 0 时的两端输入电压差。增益变小。

\section{随机失调}

随机失调是阈值电压的失配,符合正态分布。尺寸越大,失调越少。

\[A_{vt} \~ t_{ox}\sqrt{N_B}\]


\[\sigma _{\Delta VT} = \frac{A_{VT}}{\sqrt{WL}}\]

参数 \(A_{VT}\) 在尺寸变小到一定程度之后保持不变。

参数 \(K'\) 对晶体管的影响相对较小,\(W/L\) 和工艺的相关性不强。


\section{差分对的随机失调}

等效的失调电压为 

\[V_{od} = \Delta R_L \frac{I_B}{2}\]
    
\[V_{os} = \frac{\Delta R_L }{R_L}\frac{V_{gst}}{2}\]

\section{电流镜的失调}

导线电阻,弱反型区 \(V_T\) 为主要,强反型区中以 \(\beta\) 为主,面积又占主要。

\section{共模抑制比 CMRR\footnote{Common Mode Rejection Ration}}

共模增益:差模输出比共模输入

共模抑制比:差模增益比共模增益

\section{系统失调}





% End Here

\let\chapname\undefined
\ifx\mainclass\undefined
\end{document}
\fi 