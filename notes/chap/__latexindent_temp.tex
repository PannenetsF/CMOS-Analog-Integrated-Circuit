\ifx\mainclass\undefined
\documentclass[cn,11pt,chinese,black,simple]{../elegantbook}
% 微分号
\newcommand{\dd}[1]{\mathrm{d}#1}
\newcommand{\pp}[1]{\partial{}#1}

\newcommand{\homep}[1]{\section*{Problem #1}}

% FT LT ZT
\newcommand{\ft}[1]{\mathscr{F}[#1]}
\newcommand{\fta}{\xrightarrow{\mathscr{F}}}
\newcommand{\lt}[1]{\mathscr{L}[#1]}
\newcommand{\lta}{\xrightarrow{\mathscr{L}}}
\newcommand{\zt}[1]{\mathscr{Z}[#1]}
\newcommand{\zta}{\xrightarrow{\mathscr{Z}}}

% 积分求和号

\newcommand{\dsum}{\displaystyle\sum}
\newcommand{\aint}{\int_{-\infty}^{+\infty}}

% 简易图片插入
\newcommand{\qfig}[3][nolabel]{
  \begin{figure}[!htb]
      \centering
      \includegraphics[width=0.6\textwidth]{#2}
      \caption{#3}
      \label{\chapname :#1}
  \end{figure}
}

% 表格
\renewcommand\arraystretch{1.5}


% 日期

\begin{document}
\fi 
\def\chapname{01mosfet}

% Start Here
\chapter{MOSFET}

\section*{回顾}

MOS 的晶体电流公式:

线性区时: 

\[
I_d = \mu_0 C_{ox} \frac{W}{L} (V_{gs} - V_{th} - \frac{1}{2} V_{ds}) V_{ds} 
\]

饱和区时:

\[
I_d = \frac{1}{2} \mu_0 C_{ox} \frac{W}{L} (V_{gs} - V_{th})^2
\]

\section{线性区:电阻}

在线性区满足 \(V_{DS} < V_{GS} - V_{th}\),当沟道打开时,沟道高度与 \(V_{gs}\) 成比例,通过面积法理解电流中的 \(V_{ds}/2 \) 来源。

小信号的导通电阻求解为:

\[
R = \frac{\pp{V}}{\pp{I}} \approx \frac{1}{\mu_0 C_{ox} \dfrac{W}{L} (V_{gs} - V_{th}) }  
\]


电子迁移率大概是 \(\mu_n \approx 600 \text{cm}^2/Vs\), \(\mu_p \approx 250 \text{cm}^2/Vs\) 。 栅氧层电容 \(C_{ox} = \frac{\epsilon_{ox}}{t_{ox}}\) ,基本可以按照特征尺寸 \(L_{min}\) 估计栅氧厚度 \(t_{ox} = \dfrac{L_{min}}{50}\) 。一般使用 \(\text{cm}^2\) 相关的单位。定义工艺量 \(KP_{n} = \mu_n C_{ox,n}\)。

根据以上的知识可以对 MOSFET 的电阻进行快速的估算。工艺越小,由于特征尺寸小,电阻更小; PMOS 由于迁移率小,电阻更大。

\section{饱和区:放大器}

在 \(V_{ds} > V_{gs} - V_{th}\) 时,进入饱和区。对于\(I_d\)公式,如何理解其系数 \(1/2\) 以及平方项:


\[
I_d = \frac{1}{2} \mu_0 C_{ox} \frac{W}{L} (V_{gs} - V_{th})^2
\]

同样是通过沟道图理解, \(V_{ds}\) 最多造成 \(V_{gs} - V_{th}\) 的影响,沟道越长,调制效应越小。

其跨导定义如下,最后一种形式最常见,需要记忆:和 \(I_d\) 直接相关,商为能量效率。

\[
\begin{aligned}
    g_m = \frac{\pp{I_d}}{\pp{V_{gs}}} &= \mu_0 C_{ox} \frac{W}{L}(V_{gs} - V_{th}) \\
    &= \sqrt{{2 I_d \cdot \mu_0 C_{ox} \frac{W}{L}\\
    &= \frac{2 I_d}{V_{gs} - V_{th}}
\end{aligned}
\]

可以看到,跨导和漏极电流有着密切的要求,对于 \(g_m \propto \sqrt{I_d}\) 在测试中尺寸是固定的,对于 \(g_m \propto I_d\) 在设计中偏置固定。

其输出电阻 \(r_0 = V_{ds} / I_d \approx 1 / (\lambda I_d)\) 。\(\lambda = 1/(V_E \cdot L)\) , \(V_E\) 是工艺相关的量,\(L\) 是沟道长度。一般来说,\(V_{E,n} = 4 \text{V/}\mu m\),\(L = 1 \mu m\)

\subsection{单晶体管放大器}

对于共源放大器:

\[A = g_m r_0 = \frac{2 I_d}{V_{gs} - V_{th}} \cdot \frac{V_E L}{I_D} = \frac{2 V_E L}{V_{gs} - V_{th}}\]

其中 \(V_{gs} - V_{th} \approx 0.2 V\)

运算放大器的设计存在 Trade-off 如 \tabref{tab:01:1}:对于 \(g_m \approx \frac{2 I_{ds}}{V_{gs} - V_{th}}\) ,\(A = \dfrac{2 V_E L}{V_{gs} - V_{th}}\),跨导越大,速度越快。

\begin{table}[htb]
    \centering
    \caption{Trade-off}\label{tab:01:1}
    \begin{tabular}{lll}
        \hline
        & 高增益  & 高速   \\ \hline
\(V_{gs}-V_{th}\) & down & up   \\
\(L\)             & up   & down \\ \hline
    \end{tabular}
\end{table}

\section{弱反型区}

对于弱反型区有,其中 \(n > 1\) 

\[I_{d,wi} = I_{d0} \frac{W}{L} e^{\frac{V_{gs}}{nkT/q}}\]

\[g_{m,wi} = I_{d,wi} \frac{1}{n k T / q}\]

场效应管是一个水平的 BJT 三极管,漏极的反向偏置相册二极管,栅级电压的增加会降低二极管的电势壁垒,主要电流为扩散电流而不是漂移电流,导电的通道中电势几乎不变,但是离子浓度成线性变化。

\[I_{d,wi} = I_{d0} \frac{W}{L} e^{\frac{V_{gs}}{nU_T}} \cdot \left[1 - exp(-\frac{V_{ds}}{U_T})\right]\] 

其中 

\[I_{d0} = \mu_n C_{ox} (n - 1) U_t^2 e^{-V_{th}/(nV_t)}, \text{ where } U_t = \frac{kT}{q}\]

\[n = \frac{C_{ox} + C_{depl}}{C_{ox}}\approx 1.5\]

对 \(V_{ds} > 4 U_T\) 称为饱和,最后一项可以忽略。利用强反型区和弱反型区跨导相等时的电压条件,可以计算其交界点,约为\(V_{gs}-V_{th} = 2 n k T / q \approx 70 mV \),这是和工艺独立的。此时电流约为零点几个微安。保证 \(V_{gs} - V_{th} > 0.2 V\) 可以保证不同工艺中均在强反型区工作。

对于强反型区能量效率为 \(2 / (V_{gs}-V_{th})\),一般为20,弱反型区的效率几乎是一个常数。

相应的,在电压过大之后,还会进入到速度饱和区。

\section{弱反型区以及强反型区:EKV 模型}

得到的曲线在两端可以拟合

\[I(v) = K' \frac{W}{L} V_{GSTt}^2 \ln^2(1 + e^v), \text{ where } v = \frac{V_{{GST}}}{V_{GSTt}}\]

\[\text{where } V_{GSTt} = (V_{GS}-V_T)_t = 2n\frac{KT}{q}\]

定义反型系数\[i = \frac{I_{DS}}{I_{DSt}} = \ln^2(1+e^v)\]

那么\[v = \ln(e^{\sqrt{i}}-1)\]

那么\[V_{GST}=V_{GSTt} \ln(e^{\sqrt{i}}-1) \]

晶体管的最大 \(g_m/I_{DS}\)出现在弱反型区,且随着反型效率上升而下降。定义归一化效率为

\[GM/ID = \frac{g_m/I_{DS}}{(g_m/I_{DS})_{max}} = \frac{1-e^{\sqrt{i}}}{\sqrt{i}}\]

因此需要在跨导值和跨导效率进行权衡,一般取 \(200 mV\) 。

两个区之间存在一个平滑状态,一般这样定义

\[\left\{\begin{aligned}
    V_{GS} < V_{T} - 100 mV, &\text{ 弱反型}\\
    V_{T} - 100 mV < V_{GS} < V_{T} + 100 mV, &\text{ 平滑过渡}\\
    V_{GS} > V_{T} + 100 mV, &\text{ 强反型}\\
\end{aligned}\right.\]


\section{速度饱和区}

\textbf{为什么饱和区电流存在 \(V_{GS}\) 平方项?}

\(V_{GS}\) 控制两个量,一个是通道的深度,一个是两端的电压。

\textbf{为什么速度饱和区跨导却成线性?}

在速度饱和区中,电子已经以最大速度通过,电流随着两端驱动电压线性变化。将 \(C_{ox} (V_{GS} - V_T) \) 看作是导电沟道的高度。

\[I_{DS,{vs}} = W C_{ox} (V_{GS} - V_T) v_{sat}\]

其中 \(v_{sat} \approx 10^7 cm/s\) ,此时 \(g_{m,sat} = W C_{ox} v_{sat}\) 达到了最大。此时的 \(g_m/W\) 仅与物理常数有关,一般在模拟电路中不使用这个区域。

强反型区和速度饱和区分别满足跨导关系,可以得到过渡的电压,\(V_{GS} - V_{TH} \approx 0.58 V \)。

\[\left\{\begin{aligned}
    g_{m,si} &= \mu_0 C_{ox} \frac{W}{L} (V_{GS} - V_T) \\
    g_{m,sat} &= W C_{ox} v_{sat}
\end{aligned}\right\}\]

继续探索统一表示的可能,由于实际的跨导满足 \(g_m = \min (g_{m,si}, g_{m,sat})\),可以如此估算:

\[\frac{1}{g_m} = \frac{1}{g_{m,si}} + \frac{1}{g_{m,sat}}\]


其他因素:

Drain-Induced Barrier Lowering: 沟道过小时,会导致电压的改变直接作用到另一端。

Surface Scattering: 纵向电压过大时,由于栅级的反弹,会导致电子往复纵向运动,电流减小。

Impact inonize: 

\section{特征频率}

超过特征频率之后,就认为晶体管失去放大作用,一般由跨导和计生电容决定。跨导标志了驱动外部电压的能力。电容包括,氧化层电容,交叠区电容,PN结电容。

\[C_{GS} \approx\frac{2}{3} WLC_{ox}\]

\[C_{GD} = W C_{gdo}\]

达到特征频率时,\(i_{GS} = i_{DS}\) 

即 \[v_{GS} C_{GS} s = g_m v_{GS}\] 

\[C_{GS} = \frac{2}{3} WL C_{ox}\]

\[g_m = 2 K' \frac{W}{L} (V_{GS} - V_T)\]

其中 

\[K' = \frac{\mu C_{ox}}{2 n}\]


解得

\[f_T = \frac{g_m}{2 \pi C_{GS}} = \frac{1}{2 \pi} \frac{3}{2n} \frac{\mu}{L^2}(V_{GS}-V_T), \text{ 反型区}\] 

\[f_T = \frac{v_{sat}}{2 \pi L}, \text{ 饱和区}\]

\[f_m = \frac{f_T}{1 + \alpha_{BD} } ,\text{ where} \alpha_{BD} \approx \frac{C_{BD}}{C_{ox}}\]

在弱反型区:

\[GM/ID = \frac{g_m/I_{ds}}{(g_m // I_{ds})_{max}} = \frac{1 - e^{-\sqrt{i}}}{\sqrt{i}}, \text{ where } i = \frac{I_{ds}}{I_{dst}}\]

% \[\begin{aligned}
%     f_T = \frac{g_m}{2\pi C_{GS}} = &\frac{1}{2 \pi C_{gs}} \frac{I_{dst}}{nkT/q} \sqrt{i} (1 - e^\sqrt{i}) \\
%     = &\frac{2 \mu k T/q}{2 \pi L^2} \sqrt{i} (1-e^\sqrt{i})
% \end{aligned}\]

\[\begin{aligned}
    f_T = \frac{g_m}{2 \pi C_{gs}} &= \frac{1}{2 \pi C_{gs}} \frac{I_{dst}}{nkT/q}\sqrt{i}(1-e^{\sqrt{i}})\\
    &= \frac{2 \mu kT/q}{2 \pi L^2} \sqrt{i} (1 - e^{\sqrt{i}})
\end{aligned}\]

第一部分是与尺寸有关的,第二部分与工作偏置有关。


\section{总结}

设计思路:

\begin{itemize}
    \item 手工计算估计尺寸,精确设计依赖仿真
    \item 时刻牢记 \(g_m/I_D\) 曲线以及大致数值
    \item 低功耗\(V_{gs}-T_{th} < -0.1 V\),高增益\(V_{gs}-T_{th} = 0.2 V\),高速\(V_{gs}-T_{th} = 0.5 V\)
\end{itemize}

% End Here

\let\chapname\undefined
\ifx\mainclass\undefined
\end{document}
\fi 