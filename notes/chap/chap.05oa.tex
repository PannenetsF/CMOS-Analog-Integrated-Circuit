\ifx\mainclass\undefined
\documentclass[cn,11pt,chinese,black,simple]{../elegantbook}
% 微分号
\newcommand{\dd}[1]{\mathrm{d}#1}
\newcommand{\pp}[1]{\partial{}#1}

\newcommand{\homep}[1]{\section*{Problem #1}}

% FT LT ZT
\newcommand{\ft}[1]{\mathscr{F}[#1]}
\newcommand{\fta}{\xrightarrow{\mathscr{F}}}
\newcommand{\lt}[1]{\mathscr{L}[#1]}
\newcommand{\lta}{\xrightarrow{\mathscr{L}}}
\newcommand{\zt}[1]{\mathscr{Z}[#1]}
\newcommand{\zta}{\xrightarrow{\mathscr{Z}}}

% 积分求和号

\newcommand{\dsum}{\displaystyle\sum}
\newcommand{\aint}{\int_{-\infty}^{+\infty}}

% 简易图片插入
\newcommand{\qfig}[3][nolabel]{
  \begin{figure}[!htb]
      \centering
      \includegraphics[width=0.6\textwidth]{#2}
      \caption{#3}
      \label{\chapname :#1}
  \end{figure}
}

% 表格
\renewcommand\arraystretch{1.5}


% 日期

\begin{document}
\fi 
\def\chapname{05oa}

% Start Here
\chapter{运算放大器}

运放需要负反馈。

负反馈更换为电容就会转换为积分器。

增益带宽积的计算:增益带宽积为 \(GBW\) ,增益为 \(A_v = - R_2/R_1\) 


\section{极点}

单极点系统天然稳定:相移不超过 \(90 ^\circ\) 。
 
多极点系统:\(A_l = A_o / A_c\) ,当环路增益较小时,相移可以使得系统稳定。 \(70^\circ\) 相位裕度是较为安全的指标。  

\section{极点分离技术}

如何判断零点:传递函数为 0 即为零点。通过调节补偿电容 \(C_{c}\) 

第二级会比第一级的 \(g_m\) 大很多。

正零点技术:剪断前馈路径。Cascode 没有额外的成本。 串联电阻,前后阻抗相等即可。

\section{三级运放的稳定性}

新增的极点需要到五倍的带宽积之外。

\section{OTA 设计}

定义品质因素 (Figure fo Merit) 

\[FOM = \frac{GBW * C_L}{I_B}\]

寄生电容的次极点离特征频率远,影响很小。
% End Here

\let\chapname\undefined
\ifx\mainclass\undefined
\end{document}
\fi 