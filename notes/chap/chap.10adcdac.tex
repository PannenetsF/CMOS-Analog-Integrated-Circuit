\ifx\mainclass\undefined
\documentclass[cn,11pt,chinese,black,simple]{../elegantbook}
% 微分号
\newcommand{\dd}[1]{\mathrm{d}#1}
\newcommand{\pp}[1]{\partial{}#1}

\newcommand{\homep}[1]{\section*{Problem #1}}

% FT LT ZT
\newcommand{\ft}[1]{\mathscr{F}[#1]}
\newcommand{\fta}{\xrightarrow{\mathscr{F}}}
\newcommand{\lt}[1]{\mathscr{L}[#1]}
\newcommand{\lta}{\xrightarrow{\mathscr{L}}}
\newcommand{\zt}[1]{\mathscr{Z}[#1]}
\newcommand{\zta}{\xrightarrow{\mathscr{Z}}}

% 积分求和号

\newcommand{\dsum}{\displaystyle\sum}
\newcommand{\aint}{\int_{-\infty}^{+\infty}}

% 简易图片插入
\newcommand{\qfig}[3][nolabel]{
  \begin{figure}[!htb]
      \centering
      \includegraphics[width=0.6\textwidth]{#2}
      \caption{#3}
      \label{\chapname :#1}
  \end{figure}
}

% 表格
\renewcommand\arraystretch{1.5}


% 日期

\begin{document}
\fi 
\def\chapname{10}

% Start Here
\chapter{ADC \& DAC 奈奎斯特转换器} 

\section{定义}

\subsection{DAC 的分辨率} 

数字模拟的转换器的分辨率为 \(V_{REF} / 2^N\) 

\subsection{量化误差} 

量化噪声是一个白噪声。噪声功率满足 \(P_{noise} = d^2/12\) ,信号峰值满足 \(\dfrac{V_{pp}}{2}\) ,而每一份满足 \(V_{ptp} = 2^N d\)功率为 \(P_{signal} = V_{ptp}^2 / 8\) 。信噪比为 \(SNR = 6 N + 1.76 dB\) 。

\subsection{静态指标}

定义 \(DNI)\) 差分非线性度和 \(INL\) 积分非线性度。

\subsection{动态指标}

DAC 存在响应时间 \(settling time\) 。无杂散动态范围:基波与最大谐波的比值。信噪失真比。最大位数。

信号输入越大,信噪比越高。

\section{DAC} 

\subsection{电阻型}

电阻与开关太多,且电阻难匹配。

\subsection{二进制权重电阻}

对匹配要求更高,匹配精度在 6-8 位。

\subsection{阶梯型}

每个节点的等效电阻一致,在10位左右,虚地很关键,对增益有要求。

\subsection{电容型}

基本步骤是 1) 电容采样, 2) 电容放大。电荷转移到跨接到运放的电容上。

\subsection{电流转换}

转换精度的限制:电流镜。

\section{ADC} 

速度和精度的平衡。

\subsection{积分型}

放电计时。分辨率高,线性度好,复杂度低,但是速度慢。

\subsection{逐次逼近型}

ADC = DAC + compare

% End Here

\let\chapname\undefined
\ifx\mainclass\undefined
\end{document}
\fi 