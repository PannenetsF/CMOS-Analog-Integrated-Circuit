\ifx\mainclass\undefined
\documentclass[cn,11pt,chinese,black,simple]{../elegantbook}
% 微分号
\newcommand{\dd}[1]{\mathrm{d}#1}
\newcommand{\pp}[1]{\partial{}#1}

\newcommand{\homep}[1]{\section*{Problem #1}}

% FT LT ZT
\newcommand{\ft}[1]{\mathscr{F}[#1]}
\newcommand{\fta}{\xrightarrow{\mathscr{F}}}
\newcommand{\lt}[1]{\mathscr{L}[#1]}
\newcommand{\lta}{\xrightarrow{\mathscr{L}}}
\newcommand{\zt}[1]{\mathscr{Z}[#1]}
\newcommand{\zta}{\xrightarrow{\mathscr{Z}}}

% 积分求和号

\newcommand{\dsum}{\displaystyle\sum}
\newcommand{\aint}{\int_{-\infty}^{+\infty}}

% 简易图片插入
\newcommand{\qfig}[3][nolabel]{
  \begin{figure}[!htb]
      \centering
      \includegraphics[width=0.6\textwidth]{#2}
      \caption{#3}
      \label{\chapname :#1}
  \end{figure}
}

% 表格
\renewcommand\arraystretch{1.5}


% 日期

\begin{document}
\fi 
\def\chapname{CALLBACK}

% Start Here
\chapter{基本知识点}

\section{晶体管}

分在弱反型区、强反型区和速度饱和区。

核心是 \(g_m / I_{D}\) 。

特征频率是 \(f_T = \dfrac{g_m}{2 \pi C_{GS}}\) 。其中 \(C_{GS} = 2 C_{OX} / 3\) , \(C_{OX} = WL \epsilon_{ox} / t_{ox}\) , \(t_{ox} = L_{min} / 50\) 。速度饱和区的边界是 \(0.5 V\) 。

\section{模拟电路的基本构成}

增益 \(A_V = g_m r_{DS} = 2 I_{DS} / (V_{GST}) * V_E L / I_{DS} \) 。

共源共栅极,源极看,逻辑阻抗缩小了 \(A\) 倍,漏极看,源极阻抗增大了 \(A\) 倍。

二极管连接的晶体管,漏极看的电阻是 \(1 / g_m\) 。

漏极看才有可能出现高阻点。

\section{闪烁噪声}

闪烁噪声是由晶体管的面积决定的。


\section{失调}

\begin{itemize}
    \item 阈值电压
    \item 尺寸
    \item 半导体参数(较弱)
\end{itemize}

\section{频响}

相位会对一次的频响进行反复的放大。
% End Here

\let\chapname\undefined
\ifx\mainclass\undefined
\end{document}
\fi 