\ifx\mainclass\undefined
\documentclass[cn,11pt,chinese,black,simple]{../elegantbook}
% 微分号
\newcommand{\dd}[1]{\mathrm{d}#1}
\newcommand{\pp}[1]{\partial{}#1}

\newcommand{\homep}[1]{\section*{Problem #1}}

% FT LT ZT
\newcommand{\ft}[1]{\mathscr{F}[#1]}
\newcommand{\fta}{\xrightarrow{\mathscr{F}}}
\newcommand{\lt}[1]{\mathscr{L}[#1]}
\newcommand{\lta}{\xrightarrow{\mathscr{L}}}
\newcommand{\zt}[1]{\mathscr{Z}[#1]}
\newcommand{\zta}{\xrightarrow{\mathscr{Z}}}

% 积分求和号

\newcommand{\dsum}{\displaystyle\sum}
\newcommand{\aint}{\int_{-\infty}^{+\infty}}

% 简易图片插入
\newcommand{\qfig}[3][nolabel]{
  \begin{figure}[!htb]
      \centering
      \includegraphics[width=0.6\textwidth]{#2}
      \caption{#3}
      \label{\chapname :#1}
  \end{figure}
}

% 表格
\renewcommand\arraystretch{1.5}


% 日期

\begin{document}
\fi 
\def\chapname{02iccomp}

% Start Here
\chapter{模拟电路的基本构成}

运算放大器均基于差分对。如五管放大器由电流镜、差分对以及单管放大器组成。


\section{单管放大器}

对于单管放大器来说,增益为 


\[A_v = g_m r_o = \frac{2 I_D}{V_{ov}} \cdot \frac{V_E L}{I_D} = \frac{2 V_E L }{V_{ov}}\]

其中过载电压为 \(V_{ov} = V_{gs} - V_{th}\) ,对于单管来说一般增益在 100 左右(\(V_{E} L \approx 100\) , \(V_{ov} \approx 0.2 V\))。若是想要高增益,可以降低其过载电压,但是会牺牲信噪 比 \(SNR\) 与跨导 \(g_m\) ;可以增大其 \(L\) 但是会牺牲速度与面积。

如果只有大负载电容,其带宽是 

\[BW = \frac{1}{2 \pi r_{ds} C_L}\]

增益带宽积为 

\[GBW = \frac{g_m}{2 \pi C_L}\]

如果只有大输入电容,带宽为

\[BW = \frac{1}{2 \pi R_S C_{GS}}\]

增益带宽积和沟道长度 \(L\) 无关,为

\[GBW = \frac{g_m r_{ds}}{2 \pi R_S C_{GS}} = f_T \cdot \frac{r_{ds}}{R_S} \propto \frac{1}{W C_{{ox}}} \cdot \frac{1}{V_{gs} - V_{th}}\]

只有一个较大的反馈电容时,带宽为

\[BW = \frac{1}{2 \pi R_S A_{v0}} C_F\]

那么增益带宽积,和晶体管无关,只和反馈有关,为

\[GBW = \frac{1}{2 \pi R_S C_F}\]

Source Degeneration 是负反馈的形式如\figref{\chapname 1},在源极添加电阻。此时 \(V_s\) 也成为电流的影响因子。

\qfig[1]{0201.png}{源极反馈}

其反馈为 \[(V_{in} - V_s) g_m = I_{out}\] 

\[V_s = I_{out} g_m\]

输入电容发生改变:

\[C_{in} = \frac{C_{GS}}{1 + g_m R_S}\] #  参考PPT

\qfig[2]{0202.png}{源极反馈电容}

输出电阻变化,即 \(\Delta V\) 引起的 \(\Delta I\):

\[\Delta I = \frac{\Delta V_{rs}}{R_s}\]

\[R_o = r_{ds} (1 + g_m R_s) \approx A_0 R_S\] 

晶体管的噪声与失配减少 \(1 + g_m R_S\) 倍, 但是 \(R_s\) 造成额外的噪声。

\section{源随器}

是一个电压的缓冲器, \(I_B\) 为常数,\(V_{gs}\) 为常数,\(V_{out} = V_{in} \) ,那么 \(A_v = 1\) 。

由于工艺问题,PMOS 的 n 阱可以是任意电压,\(V_{gs}\) 不为常数,\(A_v < 1\)。

\section{Cascode 共源共栅级}

共源共栅级是一个共栅级的形式,作为电流的缓冲器。定义其跨阻增益

\[A_R = v_{out} / i_{in}\]

从源极向上看的电阻为 

\[R_{eq} = \frac{R_{ds} + R_l}{1 + g_m R_{ds}}\]

相比单晶体管,带宽变小,增益变大,增益带宽积不变。在低频区域提高了增益,没有更高的电流消耗。

折叠式 Cascode 和套桶式的主要参数一致,但是功耗是两倍。

调节式 Cascode 通过栅级的调控将原来的衰减一个放大系数变为两个。

\section{电流镜}

理想电流镜是由二极管连接晶体管和单管放大器组成的。

对于二极管,输入电阻为 \(1/g_m\) ,带宽为 \(g_m / (2 \pi (C_{gs} + C_{ds})) \approx f_T / 2\)

电流镜的输出阻抗是单晶体管的输出阻抗,输出精度可以描述为 

\[\frac{\Delta I}{I}\]

改进电流镜:输出阻抗造成输入输出电流不相等,电压余度消耗过多。

低电压电流镜需要和运算放大器进行配合,开销很大。输出端的最小值仅有 0.2 V 时,不能使用 Cascode 型技术。虽然没有改变输出电阻,但是改善了系统性的失配问题。

超低电压电流镜\footnote{判断是否是 Cascode :是否均工作在饱和区}:电流晶体管在线性区,但是通过运放锁定电压。

电流镜在高频时,由于倍数关系有
\begin{equation}\begin{aligned}
    C_{G} &=(1+B) C_{G S}+C_{D S 1} \\
    B W &=\frac{g_{m}}{2 \pi\left(C_{G}+C_{D S 1}\right)} \\
    & \approx f_{T} \frac{1}{(2+B)}
\end{aligned}\end{equation}

存在米勒效应。

\section{差分对}

差分对的本质仍然是电压输入电流输出,能效降低为单晶体管的一半。

\begin{equation}\frac{ i _{ O d }}{ I _{ B }}=\frac{ v _{ ld }}{\left( V _{ GS }- V _{ T }\right)}  \sqrt{1-\frac{1}{4}\left(\frac{ v _{ ld }}{ v _{ GS }- V _{ T }}\right)^{2}}\end{equation}

其中 \(V_{ld}\) 是差分输出电压,\(i_{od}\) 是输出电流, \(I_B\) 是总的直流电流。

差分放大器相当于差分对 + 负载。

交叉耦合对:等效差分输入电阻为 \(-1/g_m\) ,

% End Here

\let\chapname\undefined
\ifx\mainclass\undefined
\end{document}
\fi 